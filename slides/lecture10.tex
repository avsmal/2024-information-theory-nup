% !TeX spellcheck = en_US
\documentclass[aspectratio=169]{beamer}

\usetheme{default}

\setbeamertemplate{navigation symbols}{}

\usepackage{tikz}

\usepackage{amsmath,amssymb}

\newcommand{\pitem}{\pause\item}

\newtheorem{statement}{Statement}

\newcommand{\bits}{\{0,1\}}
\newcommand{\bitstr}{\bits^*}
\newcommand{\sshalf}{{\textstyle\frac12}}
\newcommand{\seqn}[2]{{#1}_1,{#1}_2,\dotsc,{#1}_{#2}}
\newcommand{\seqin}[3]{{#1}_{{#2}_1},{#1}_{{#2}_2},\dotsc,{#1}_{{#2}_{#3}}}
\newcommand{\IC}{\mathrm{IC}}
\newcommand{\poly}{\mathrm{poly}}
\newcommand{\Nat}{\mathbb{N}}

\DeclareMathOperator{\dom}{dom}
\DeclareMathOperator{\rank}{rank}
\DeclareMathOperator{\rng}{rng}
\DeclareMathOperator*{\E}{\mathbb{E}}

\DeclareMathOperator{\KL}{D_{KL}}
\DeclareMathOperator{\CE}{CE}
\newcommand{\midd}{\parallel}


\newenvironment{sol}{%
    \begin{block}{}%
        \setbeamercolor{block body}{bg=lightgray}%
    }{
\end{block}}

\AtBeginSection[]{
    \begin{frame}
        \vfill
        \centering
        \begin{beamercolorbox}[sep=8pt,center,shadow=true,rounded=true]{title}
            \usebeamerfont{title}\insertsectionhead\par%
        \end{beamercolorbox}
        \vfill
    \end{frame}
}


\title{Kolmogorov Complexity and Randomness}
\author{Alexander Smal at NUP}
\date{09.01.2025.\\ Lecture 10}

\begin{document}
%%%%%%%%%%%%%%%%%%%%%%%%%%%%%%%%%%%%%%%%%%%%%%%%%%%%%%%%%%%%%%%%%%%%%%%%
\begin{frame}[plain]
    \maketitle
\end{frame}
%%%%%%%%%%%%%%%%%%%%%%%%%%%%%%%%%%%%%%%%%%%%%%%%%%%%%%%%%%%%%%%%%%%%%%%%

%%%%%%%%%%%%%%%%%%%%%%%%%%%%%%%%%%%%%%%%%%%%%%%%%%%%%%%%%%%%%%%%%%%%%%%%
\begin{frame}{Definition of Randomness}
Consider infinite a sequence $\bar{x} = x_1 x_2 x_3 \dotso x_n \dotso$. \\\pause
What properties do we expect from a random $\bar x$?\\\pause
A natural definition would be of the form $\forall n,\ K(x_1 x_2 x_3 \dotso x_n) \ge n - c$.
\medskip\pause

    For any sequence $\bar{x} = x_1 x_2 x_3 \dotso x_n \dotso$, there exists an $n$ such that
    \[
    K(x_1, \dotsc, x_n) \le n - \log n + O(1).
    \]
    (i.e., for any $c$, there exists a prefix such that $K(x_1, \dotsc, x_n) \le n - c$).
\medskip\pause

    Take some prefix of length $k$ and interpret it as the binary representation of some number $m$ (adding a leading one), and consider its continuation of length $m$:
    \[
    \underbrace{1 x_1 x_2 x_3 \dotso x_k}_{\overline{m}}
    \underbrace{x_{k+1} \dotso x_{k + m}}_y,
    \]
    where $|y| = m$. Let $n = m + k$. Then it is claimed that
    \[
    K(x_1 \dotso x_{m+k}) \le K(y) + O(1) \le m + O(1) \le n - k + O(1) \le n - \log n + O(1).
    \]
    Knowing the string $y$, one can determine $m = |y|$ and output $\overline{m}y$ w/o leading 1.
\end{frame}
%%%%%%%%%%%%%%%%%%%%%%%%%%%%%%%%%%%%%%%%%%%%%%%%%%%%%%%%%%%%%%%%%%%%%%%%


%%%%%%%%%%%%%%%%%%%%%%%%%%%%%%%%%%%%%%%%%%%%%%%%%%%%%%%%%%%%%%%%%%%%%%%%
\begin{frame}{Prefix Kolmogorov Complexity}
    \begin{definition}
        \emph{Prefix complexity of $x$ relative to $F$}:
        \[
        KP_F(x) = \min \{|p| : F(p) = x\},
        \]
        where $F$ is a function defined on a (non)prefix domain, i.e., if $F(p_1)$ and $F(p_2)$ are defined, then $p_1 \not\sqsubset p_2$.
    \end{definition}
    \begin{definition}
        A non-prefix method of describing $F$ is \emph{not worse} than a non-prefix method of describing $G$, denoted $F \prec G$, if $\exists c \, \forall x, \, KP_F(x) \le KP_G(x) + c$.
    \end{definition}
    \begin{theorem}
        There exists an optimal method of non-prefix description $UP$.
    \end{theorem}

    \begin{definition}
        $KP(x) = KP_{UP}(x)$ is called \emph{prefix Kolmogorov complexity of $x$}.
    \end{definition}

\end{frame}
%%%%%%%%%%%%%%%%%%%%%%%%%%%%%%%%%%%%%%%%%%%%%%%%%%%%%%%%%%%%%%%%%%%%%%%%

%%%%%%%%%%%%%%%%%%%%%%%%%%%%%%%%%%%%%%%%%%%%%%%%%%%%%%%%%%%%%%%%%%%%%%%%
\begin{frame}{Proof of Theorem}
    \textbf{Problem:} not all programs have a non-prefix domain of definition. \\\medskip

    One can transform any program $\pi_i$ into a program with a non-prefix domain $\pi'_i$:
    \begin{itemize}
        \item if $\pi_i$ computes a function $F_i$ with a non-prefix domain, then $\pi'_i$ also computes $F_i$,
        \item if $\pi_i$ computes something else, then $\pi'_i$ computes some function with a non-prefix domain (the domain can be empty).
    \end{itemize}
    After this, we will use a construction similar to Solomonov-Kolmogorov Theorem:
    $UP(\underbrace{11\dotso 1}_n 0p) = \pi_n'(p)$.

    The program $\pi_n'$ on input $p$ runs program $\pi_n$ in parallel on all inputs:
    \[\pi_n(0), \pi_n(1), \pi_n(00), \pi_n(01), \dotsc, \pi_n(p), \dotsc\]
    If at any point it discovers that $\pi_n$ has a non-prefix domain, then $\pi'(p)$ enters a loop.
    If at any point $\pi(p)$ halts and no violation of non-prefixity has been detected, then $\pi'(p) = \pi(p)$.
\end{frame}
%%%%%%%%%%%%%%%%%%%%%%%%%%%%%%%%%%%%%%%%%%%%%%%%%%%%%%%%%%%%%%%%%%%%%%%%

%%%%%%%%%%%%%%%%%%%%%%%%%%%%%%%%%%%%%%%%%%%%%%%%%%%%%%%%%%%%%%%%%%%%%%%%
\begin{frame}{Properties of Prefix Kolmogorov Complexity}
\begin{statement}
    The prefix Kolmogorov complexity has the following properties:
    \begin{itemize}
        \item $KP(x,y) \le KP(x) + KP(y) + O(1)$.
        \item $KP(x) \le K(x) + 2\log K(x) + O(1)$.
        \item $\sum_{x \in \{0,1\}^k} 2^{-KP(x)} \le 1$.
    \end{itemize}
\end{statement}
\begin{proof}\mbox{}
    \begin{itemize}
        \item As the programs are prefix-free now, we can just concatenate it.
        \item The term $2\log K(x)$ arises from the transformation of the string $p$ into a non-prefix string $p' = \overline{\overline{|p|}}01p$.
        \item Similarly to the Kraft-MacMillan inequality for prefix codes.
    \end{itemize}
\end{proof}

\end{frame}

%%%%%%%%%%%%%%%%%%%%%%%%%%%%%%%%%%%%%%%%%%%%%%%%%%%%%%%%%%%%%%%%%%%%%%%%
\begin{frame}{Martin-Löf Random Sequences}
    A sequence $\bar{x} = x_1 x_2 \dotso x_n \dotso$ is \emph{Martin-Löf random} if $\exists c \, \forall n, \, KP(x_1 \dotso x_n) \ge n - c$.
    \pause
    \begin{theorem}
        Almost all sequences $\bar{x} = x_1 x_2 \dotso x_n \dotso$ are Martin-Löf random, i.e., non-random sequences have measure 0 according to Bernoulli measure.
    \end{theorem}
    \pause
    \begin{statement}
        The following properties hold for Martin-Löf random sequences:
        \begin{itemize}
            \item Every Martin-Löf random sequence is non-computable.
            \item If $\bar{x}$ is Martin-Löf random, then
            $
            \lim_{n\to\infty} \frac{\text{number of ones}}{n} = \frac{1}{2}.
            $
        \end{itemize}
    \end{statement}
    \pause
    \begin{theorem}[Law of Large Numbers in Hardy-Littlewood Form]
        For almost all sequences $\bar{x} = x_1 x_2 \dotso x_n \dotso$ (with probability 1)
        \[
        \left|\frac{x_1 + x_2 + \dotsb + x_n}{n} - \frac{1}{2}\right| =
        O\left(\sqrt{\frac{\ln n}{n}}\right).
        \]
    \end{theorem}
\end{frame}
%%%%%%%%%%%%%%%%%%%%%%%%%%%%%%%%%%%%%%%%%%%%%%%%%%%%%%%%%%%%%%%%%%%%%%%%


%%%%%%%%%%%%%%%%%%%%%%%%%%%%%%%%%%%%%%%%%%%%%%%%%%%%%%%%%%%%%%%%%%%%%%%%
\begin{frame}{Method of Incompressible Objects}
    \begin{definition}
        \emph{Finite automaton with multiple heads}: transition function is based on
        \begin{itemize}
            \item the internal state of the automaton,
            \item the symbols under the heads,
        \end{itemize}
        returns
        \begin{itemize}
            \item the next state,
            \item and the indices of the heads to be moved,
            \item with at least one head moving at each step.
        \end{itemize}
    \end{definition}

    Class $\mathcal{L}_k$ is the class of languages recognized by finite automata with $k$ heads.

    \begin{theorem}\label{thm:kDFA}
        $\mathcal{L}_k \subsetneq \mathcal{L}_{k+1}$.
    \end{theorem}

\end{frame}
%%%%%%%%%%%%%%%%%%%%%%%%%%%%%%%%%%%%%%%%%%%%%%%%%%%%%%%%%%%%%%%%%%%%%%%%


%%%%%%%%%%%%%%%%%%%%%%%%%%%%%%%%%%%%%%%%%%%%%%%%%%%%%%%%%%%%%%%%%%%%%%%%
\begin{frame}[fragile]{Hard Language for $k$-head Automaton}
    Consider the following family of languages over the alphabet $\{0,1,\#\}$
    \[
    A_n = \{w_1\#w_2\#\dotsb\#w_n\#w_n\#\dotsb\#w_1 \mid w_i \in \bitstr\},
    \]
    where $w_i \in \bitstr$, for all $i \in \{1,2,\dotsc,n\}$.

    For $n = 1$, the language $A_1 = \{w_1\#w_1\}$ requires two heads.\pause

    For $n = 3$, it can be recognized with our heads:
    \[
    \underset{\fbox{1}}{w_1}\#w_2\#w_3\#\underset{\fbox{2}}{w_3}\#\underset{\fbox{3}}{w_2}\#\underset{\fbox{4}}{w_1}.
    \]\pause
    But it can also be done with three heads (can you find a trick?):\pause
    \[
    \underset{\fbox{1}}{w_1}\#w_2\#\underset{\fbox{2}}{w_3}\#\underset{\fbox{3}}{w_3}\#w_2\#w_1.
    \]\pause

    Using the trick $k$ heads can recognize the language $A_n$ for $n \le (k-1) + \dotsb + 1$.
    \begin{lemma}
        $A_n$ is not recognized by a finite automaton with $k$ heads if $n > \frac{k \cdot (k-1)}{2}$.
    \end{lemma}
\end{frame}
%%%%%%%%%%%%%%%%%%%%%%%%%%%%%%%%%%%%%%%%%%%%%%%%%%%%%%%%%%%%%%%%%%%%%%%%


%%%%%%%%%%%%%%%%%%%%%%%%%%%%%%%%%%%%%%%%%%%%%%%%%%%%%%%%%%%%%%%%%%%%%%%%
\begin{frame}{Proof: Protocol for Automaton}
    A pair of heads $(i,j)$ \emph{inspects} $w_\ell$ if there is a step when the $i$-th head reads a symbol from the left copy of $w_\ell$ and the $j$-th head reads a symbol from the right copy of $w_\ell$.\medskip\pause

    For any $x \in A_n$ and for any pair $(i,j)$, there exists at most one block $w_\ell$ such that the pair $(i,j)$ inspects $w_\ell$. \medskip\pause

    If $n > k \cdot (k-1)/2$, then there will be a block that is not inspected by any pair of heads. Let us consider some $x \in A_n$ and assume that the block $w_\ell$ is not inspected.\medskip\pause

    The block $w_\ell$ is not inspected, so as long as some heads are in the left copy of $w_\ell$, no heads can be in the right copy of $w_\ell$.\medskip\pause

    Lets write down the \emph{protocol for the automaton on the word $x$ with parameter $\ell$}.\\
    We record the state of the automaton each time the following events occur:
    \begin{itemize}
        \item entry of a head into the copy of $w_\ell$,
        \item exit of a head from the copy of $w_\ell$.
    \end{itemize}
    The state of the automaton will be described by its internal state and the positions of all heads. We denote such a protocol by $\pi(x,\ell)$.

\end{frame}
%%%%%%%%%%%%%%%%%%%%%%%%%%%%%%%%%%%%%%%%%%%%%%%%%%%%%%%%%%%%%%%%%%%%%%%%

%%%%%%%%%%%%%%%%%%%%%%%%%%%%%%%%%%%%%%%%%%%%%%%%%%%%%%%%%%%%%%%%%%%%%%%%
\begin{frame}{Proof: Every Block is Inspected}
    Suppose that for a specific
\[
    x = w_1\#w_2\#\dotsb\#w_\ell\#\dotsb\#w_n\#w_n\#\dotsb\#w_\ell\#\dotsb\#w_1,
\]
the finite automaton does not inspect the block $\ell$.

Consider the input $x'$ with a different block $w'_\ell$:
\[
x' = w_1\#w_2\#\dotsb\#\alert{w'_\ell}\#\dotsb\#w_n\#w_n\#\dotsb\#\alert{w'_\ell}\#\dotsb\#w_1.
\]

If the protocols are equal, then the automaton must also accept the input
\[
x'' = w_1\#w_2\#\dotsb\#w_\ell\#\dotsb\#w_n\#w_n\#\dotsb\#\alert{w'_\ell}\#\dotsb\#w_1.
\]
If some heads are in $w_\ell$, then the automaton operates on $x''$ as it does on input $x$. If some heads are in $w'_\ell$, then the automaton operates as it does on input $x'$. Therefore, it must accept $x'' \not\in A_n$. This leads us to a contradiction.\medskip

    The language $A_{\frac{k \cdot (k + 1)}{2}}$ belongs to $\mathcal{L}_{k+1}$ but does not belong to $\mathcal{L}_k$.
\end{frame}
%%%%%%%%%%%%%%%%%%%%%%%%%%%%%%%%%%%%%%%%%%%%%%%%%%%%%%%%%%%%%%%%%%%%%%%%

\end{document}
