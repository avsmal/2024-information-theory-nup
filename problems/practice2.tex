\documentclass[a4paper]{article}
\usepackage[empty]{fullpage}
\usepackage{cmap}
\usepackage[T2A]{fontenc}
\usepackage[utf8]{inputenc}
\usepackage[english]{babel}
\usepackage{amssymb,amsmath,amsfonts}
\usepackage{csquotes}
\usepackage{titling}
\usepackage[useregional]{datetime2}
\usepackage{graphicx}
\usepackage{float}

\title{Practice 2: Shannon Entropy}
\date{\DTMdate{2024-10-16}}

\preauthor{}
\postauthor{}
\author{}
\setlength{\droptitle}{-50pt}

\begin{document}

\maketitle

\section*{Entropy}

\begin{enumerate}
  \item
  Prove that Shannon entropy is not less than the minimum entropy, defined as \(H_{min} = \min_{i}(-\log{p_i})\).

  \item Let the probabilities of the outcomes of a random variable be \(1/2\),
  \(1/4\), \(1/8\), \ldots, \(1/2^n\), \(1/2^n\). What does its entropy approach as \(n \to \infty\)?
  The same question for a random variable with outcome probabilities \(1/3\),
  \(1/3\), \(1/9\), \(1/9\), \ldots, \(1/3^n\), \(1/3^n\), \(1/3^n\).

  \item Prove the formula  
  \[
  h(p_1,\dots,p_n,q_1,\dots,q_m) = h(p_1+\dots +p_n,q_1+\dots +q_m) + (p_1+\dots +p_n)h(p'_1, \dots, p'_n) + (q_1+\dots +q_m)h(q'_1, \dots, q'_m)
  \]
  where \(p'_i = \frac{p_i}{p_1+\dots +p_n}\) and \(q'_i = \frac{q_i}{q_1+\dots +q_m}\).

  % \item Show that the quantity \(H(\zeta|A)\) can be both greater and less than \(H(\zeta)\).

  \item Prove that \(I(f(\alpha):\beta) \leq I(\alpha:\beta)\) for any function \(f\).

  \item Prove that the quantities \(\alpha, \beta, \gamma\) are independent jointly (i.e., when the probability of the event \((\alpha=\alpha_i, \beta=\beta_j, \gamma=\gamma_k)\) equals the product of the three individual probabilities) if and only if
  \[
  H(\alpha,\beta, \gamma) = H(\alpha) + H(\beta) + H(\gamma).
  \]

  \item Prove that \(I((\alpha,\beta):\gamma) \geq I(\alpha : \gamma)\).

  \item Prove that 
  \[
  I((\alpha,\beta):\gamma) = I(\alpha : \gamma) + I(\beta:\gamma\mid\alpha).
  \]

  \item Prove that if \(I(\alpha:\gamma|\beta)=0\), then \(I(\alpha:\gamma) \leq I(\alpha:\beta)\), and thus \(I(\alpha:\gamma) \leq H(\beta)\).

  \item Prove that \(I((\alpha,\beta):\gamma) \geq I(\alpha:\gamma)\) and that the difference between the left and right sides equals \(I(\beta:\gamma|\alpha)\).

  \item Prove the inequality 
  \[
  2H(\alpha,\beta,\gamma) \leq H(\alpha, \beta) + H(\alpha, \gamma) + H(\beta,\gamma).
  \]

  % \item (Shearer's inequality) Prove the following generalization of the previous inequality.
  % Let \(T_1, \ldots , T_k\) be arbitrary tuples composed of the variables \(\alpha_1, \ldots, \alpha_n\), with each variable appearing in exactly \(r\) tuples. Then \(r H(\alpha_1,\dots, \alpha_n) \leq H(T_1) + H(T_2) + \cdots + H(T_k)\).

\end{enumerate}

\end{document} 
