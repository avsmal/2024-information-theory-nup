\documentclass[a4paper,14pt]{extarticle}
%\usepackage[empty]{fullpage}
\usepackage{cmap}
\usepackage[utf8]{inputenc}
\usepackage{amssymb,amsmath,amsfonts}
\usepackage{titling}
\usepackage[useregional]{datetime2}
\usepackage{graphicx}
\usepackage{float}
\usepackage{tikz}

\pagenumbering{gobble}

\newcommand*{\DKL}{\ensuremath{D_{\mathrm{KL}}}}

\title{Midterm}
\date{20.11.2024}

\preauthor{}
\postauthor{}
\author{}
\setlength{\droptitle}{-50pt}

\newcommand{\bits}{\{0,1\}}
\newcommand{\bitstr}{\bits^*}
\newcommand{\sshalf}{{\textstyle\frac12}}
\newcommand{\seqn}[2]{{#1}_1,{#1}_2,\dotsc,{#1}_{#2}}
\newcommand{\seqin}[3]{{#1}_{{#2}_1},{#1}_{{#2}_2},\dotsc,{#1}_{{#2}_{#3}}}
\newcommand{\IC}{\mathrm{IC}}
\newcommand{\poly}{\mathrm{poly}}
\newcommand{\Nat}{\mathbb{N}}

\begin{document}

    \maketitle

    \section*{Midterm}

    \begin{enumerate}
        \item Let the random variable $\alpha$ have the distribution $\frac{1}{3}, \frac{2}{3}$, and the random variable $\beta$ have the distribution $\frac{1}{2}, \frac{1}{2}$. Within what limits can $I(\beta:\alpha)$, $H(\alpha \mid \beta)$, $H(\beta \mid \alpha)$, and $H(\alpha, \beta)$ vary?

        \item Construct jointly distributed random variables $\xi, \eta, \beta, \gamma$, for which the inequality $$ I(\xi : \eta) \leq I(\xi : \eta \mid \beta) + I(\xi : \eta \mid \gamma) + I(\beta : \gamma) $$ does not hold.

        \item Vasya chooses a number $x \in \{1, 2, \dots, n\}$ randomly such that the probability of choosing the number $i$ is proportional to $\frac{1}{i^2}$. Prove that Kolya can, by asking Vasya on average $O(1)$ yes/no questions, determine the chosen number.

        \item Alice is given the value of the random variable $\alpha$, while Bob is given the value of some function $f$ of $\alpha$. Invent an algorithm that allows Alice to communicate the value of $\alpha$ to Bob, transmitting on average no more than $H(\alpha \mid f(\alpha)) + 1$ bits.

        \item Alena chooses a number $x \in \{1, 2, \dots, n\}$ randomly according to a known probability distribution by Valera. It is known that Valera can learn the chosen number by asking Alena on average $q$ yes/no questions. Prove that some number is then chosen by Alena with probability no less than $2^{-q}$.


    \end{enumerate}

\end{document}
