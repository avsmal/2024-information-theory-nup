% !TeX spellcheck = en_US
\documentclass[aspectratio=169]{beamer}
\usetheme{default}

\setbeamertemplate{navigation symbols}{}

\usepackage{amsmath,amssymb}


\newcommand{\pitem}{\pause\item}

\newcommand{\bits}{\{0,1\}}
\newcommand{\bitstr}{\bits^*}
\newcommand{\sshalf}{{\textstyle\frac12}}
\newcommand{\seqn}[2]{{#1}_1,{#1}_2,\dotsc,{#1}_{#2}}
\newcommand{\seqin}[3]{{#1}_{{#2}_1},{#1}_{{#2}_2},\dotsc,{#1}_{{#2}_{#3}}}
\newcommand{\IC}{\mathrm{IC}}
\newcommand{\poly}{\mathrm{poly}}
\newcommand{\Nat}{\mathbb{N}}


\newenvironment{sol}{%
\begin{block}{}%
\setbeamercolor{block body}{bg=lightgray}%
}{
\end{block}}

 \AtBeginSection[]{
       \begin{frame}
           \vfill
           \centering
           \begin{beamercolorbox}[sep=8pt,center,shadow=true,rounded=true]{title}
                 \usebeamerfont{title}\insertsectionhead\par%
               \end{beamercolorbox}
           \vfill
           \end{frame}
     }



\title{Information Theory}
\author{Alexander Smal at NUP}
\date{09.10.2024.\\ Lecture 1}
\begin{document}
%%%%%%%%%%%%%%%%%%%%%%%%%%%%%%%%%%%%%%%%%%%%%%%%%%%%%%%%%%%%%%%%%%%%%%%%
\begin{frame}[plain]
    \maketitle
\end{frame}
%%%%%%%%%%%%%%%%%%%%%%%%%%%%%%%%%%%%%%%%%%%%%%%%%%%%%%%%%%%%%%%%%%%%%%%%

%%%%%%%%%%%%%%%%%%%%%%%%%%%%%%%%%%%%%%%%%%%%%%%%%%%%%%%%%%%%%%%%%%%%%%%%
\begin{frame}
    \frametitle{About the course}
    Central question of this course:\medskip

    \begin{quote}
        How to measure the amount of information?
    \end{quote}

    \medskip\pause
    We are going to consider tree approaches:
    \begin{enumerate}
        \item \textbf{Combinatorial.} Hartley's formula.
        \item \textbf{Probabilistic.} Shannon's entropy.
        \item \textbf{Algorithmic.} Kolmogorov complexity.
    \end{enumerate}
    \medskip\pause
    We will see applications in:
    \begin{itemize}
        \item coding theory,
        \item cryptography,
        \item algorithms,
        \item \ldots
    \end{itemize}
\end{frame}

%%%%%%%%%%%%%%%%%%%%%%%%%%%%%%%%%%%%%%%%%%%%%%%%%%%%%%%%%%%%%%%%%%%%%%%%
\section{Combinatorial approach}
%%%%%%%%%%%%%%%%%%%%%%%%%%%%%%%%%%%%%%%%%%%%%%%%%%%%%%%%%%%%%%%%%%%%%%%%
\begin{frame}{Hartley's formula}
Let \(A\) be a finite set.
\begin{definition}[Ralph Hartley, 1928]
     \emph{The amount of information in \(A\)} is \(\chi(A) = \log_{10}|A|\).
\end{definition}

\pause

\begin{block}{Remarks}
\begin{itemize}
    \item A set of size 10 has 1 unit of information.

    \item This unit has several names: \emph{hartley} (\emph{Hart}), \emph{ban}, \emph{dit}.

    \item Replacing $\log_{10}$ with $\log_2$ will give us \emph{shannons} (\emph{Sh}).

    \item Replacing $\log_{10}$ with $\ln$ will give us \emph{nats}.

    \item 1 nat is equal to $1/\ln 2 \approx 1.44$ Sh and $1/\ln 10 \approx 0.434$ Hart.

    \item In this course we will measure the amount of information in shannons and for simplicity call it bits.

\end{itemize}
\end{block}
\end{frame}
%%%%%%%%%%%%%%%%%%%%%%%%%%%%%%%%%%%%%%%%%%%%%%%%%%%%%%%%%%%%%%%%%%%%%%%%

%%%%%%%%%%%%%%%%%%%%%%%%%%%%%%%%%%%%%%%%%%%%%%%%%%%%%%%%%%%%%%%%%%%%%%%%
\begin{frame}{Hartley's formula. Application}
\begin{example}
Suppose that we know that $x\in A$. How much information is contained in the fact that $x\in B$?
\end{example}
\pause
\begin{solution}
\(\chi(A) - \chi(A\cap B)\)
\end{solution}
\pause
\begin{example}
    The ordering of the set $\{\seqn{a}{5}\}$ is unknown. We have learned that \(a_1>a_2\) or \(a_3>a_4\). How many bits of information have we gained?
\end{example}
\pause
\begin{solution}
    The set \(A\) consists of \(5!\) permutations, while the set \(B\) consists of the permutations that satisfy the new conditions. It is easy to verify that \(|B| = 90\).
    \\Thus, we have learned \(\log 120 - \log 90 = \log(4/3)\) bits.
\end{solution}
\end{frame}
%%%%%%%%%%%%%%%%%%%%%%%%%%%%%%%%%%%%%%%%%%%%%%%%%%%%%%%%%%%%%%%%%%%%%%%%

%%%%%%%%%%%%%%%%%%%%%%%%%%%%%%%%%%%%%%%%%%%%%%%%%%%%%%%%%%%%%%%%%%%%%%%%
\begin{frame}{Properties}
Let \(A\subseteq\bitstr\times\bitstr\). Denote by \(\pi_1(A)\) and \(\pi_2(A)\) the projections of the set \(A\) onto the first and second coordinates, respectively, and let \(\chi_1(A) = \log|\pi_1(A)|\) and \(\chi_2(A) = \log|\pi_2(A)|\).

\begin{theorem}
    \(\chi(A) \le \chi_1(A) + \chi_2(A)\).
\end{theorem}

\begin{definition}
    The amount of information in the second coordinate of \(A\subset\bitstr\times\bitstr\) given the first coordinate is
    \[\chi_{2|1} = \log\max_{a\in\pi_1(A)}\bigl|\{x \mid (a, x)\in A\}\bigr|.\]
\end{definition}\vspace{-5mm}
\begin{theorem}
    \(\chi(A) \le \chi_1(A) + \chi_{2|1}(A)\).
\end{theorem}
\end{frame}
%%%%%%%%%%%%%%%%%%%%%%%%%%%%%%%%%%%%%%%%%%%%%%%%%%%%%%%%%%%%%%%%%%%%%%%%

%%%%%%%%%%%%%%%%%%%%%%%%%%%%%%%%%%%%%%%%%%%%%%%%%%%%%%%%%%%%%%%%%%%%%%%%
\begin{frame}{Properties (contd)}
\begin{theorem}\label{thm:volume}
    For \(A\subset\bitstr\times\bitstr\times\bitstr\)
    \[2\cdot\chi(A) \le \chi_{12}(A) + \chi_{13}(A) + \chi_{23}(A).\]
\end{theorem}
\begin{corollary}
    The square of the volume of a three-dimensional body does not exceed the product of the areas of its projections onto the coordinate planes.
\end{corollary}

\begin{theorem}
    If \(f: X\to Y\)
    \begin{enumerate}
        \item is a surjection, then \(\chi(Y)\le \chi(X)\),
        \item is an injection, then \(\chi(X)\le \chi(Y)\).
    \end{enumerate}
\end{theorem}
\end{frame}

%%%%%%%%%%%%%%%%%%%%%%%%%%%%%%%%%%%%%%%%%%%%%%%%%%%%%%%%%%%%%%%%%%%%%%%%
\begin{frame}{Application: The Game of 10 Questions}
    How many yes/no questions needed to determine a number from 1 to \(N\), if (a) questions can be asked adaptively; (b) questions must be written down in advance.

    \pause
    \begin{block}{Upper bound}
    The upper bound \(\lceil\log N\rceil\) in both cases.
    \end{block}
    \pause

    \begin{block}{Lower bound}
    Let \(A=[N]\). The set \(Q = \{(\seqn{q}{k})\}\) is the set of protocols (answers to the questions). Consider \(A\) and \(Q\) as projections of some set \(S\) of all outcomes of the game onto different coordinates. \\
    Then the following inequalities hold:
    \begin{itemize}
        \item \( \chi_Q(S) = \chi(Q) \le \chi_1(Q) + \chi_2(Q) + \dotsb + \chi_k(Q) \le k, \)
        \item \( \chi_A(S) = \chi(A) \le \chi(S) \le \chi_Q(S) + \chi_{A|Q}(S) \le k + 0 = k. \)
    \end{itemize}
    Thus, we get \(\log N = \chi(A) \le k\).
    \end{block}
\end{frame}
%%%%%%%%%%%%%%%%%%%%%%%%%%%%%%%%%%%%%%%%%%%%%%%%%%%%%%%%%%%%%%%%%%%%%%%%

%%%%%%%%%%%%%%%%%%%%%%%%%%%%%%%%%%%%%%%%%%%%%%%%%%%%%%%%%%%%%%%%%%%%%%%%
\begin{frame}{Cost of Information}
    Let a secret integer be chosen from 1 to \(n\) (where \(n \ge 2\)).
    Any yes/no questions can be asked. For a positive answer, we will pay 1 euro, and for a negative answer—two euros. How much it is necessary and sufficient to pay to guess the number?
\end{frame}
%%%%%%%%%%%%%%%%%%%%%%%%%%%%%%%%%%%%%%%%%%%%%%%%%%%%%%%%%%%%%%%%%%%%%%%%

%%%%%%%%%%%%%%%%%%%%%%%%%%%%%%%%%%%%%%%%%%%%%%%%%%%%%%%%%%%%%%%%%%%%%%%%
\begin{frame}{Sorting stones}
    \begin{example}
    How many comparisons are needed to sort \(N\) stones by weight?
    \end{example}

    \begin{solution}
    \textbf{Lower Bound.} It will require \(\lceil\chi(S_N)\rceil = \lceil\log n!\rceil\) comparisons.
    \pause\medskip

    \textbf{Upper Bound.} We will sort using insertion sort with binary search for the insertion point. The number of comparisons is:
    \[
    \lceil\log 2\rceil + \lceil\log 3\rceil + \dotsb + \lceil\log n\rceil \le \log n! + n - 1 = n\log n + O(n).
    \]
    \end{solution}

    \pause

    How many weighings are needed to sort \(N\) stones by weight?
    Find the exact answer to this question for \(N = 2, 3, 4, 5\).

    \pause
    \emph{Hint: use a greedy strategy where each weighing provides maximum information.}
\end{frame}
%%%%%%%%%%%%%%%%%%%%%%%%%%%%%%%%%%%%%%%%%%%%%%%%%%%%%%%%%%%%%%%%%%%%%%%%

%%%%%%%%%%%%%%%%%%%%%%%%%%%%%%%%%%%%%%%%%%%%%%%%%%%%%%%%%%%%%%%%%%%%%%%%
\begin{frame}{Application: Finding the Fake Coin}
    \begin{itemize}
        \item There are 20 visually identical coins, one of which is fake and lighter than the others.
        How to find the fake coin using a balance scale? What is the minimum number of weighings needed?

        \pitem There are 13 visually identical coins, one of which is fake (with unknown relative weight).
        Can you find the fake coin and determine its relative weight in three weighings?

        \pitem There are 15 visually identical coins, one of which is fake. Can you find the fake coin in three weighings
        if you do not need to determine its relative weight?

        \pitem There are 14 visually identical coins, one of which is fake. Can you find the fake coin in three weighings
        if you do not need to determine its relative weight?

        \pitem Find one fake coin among 12 in three weighings, if its relative weight is unknown. Hint: use a "greedy" strategy, where each weighing provides maximum information.
    \end{itemize}
\end{frame}
%%%%%%%%%%%%%%%%%%%%%%%%%%%%%%%%%%%%%%%%%%%%%%%%%%%%%%%%%%%%%%%%%%%%%%%%

\end{document}

%%%%%%%%%%%%%%%%%%%%%%%%%%%%%%%%%%%%%%%%%%%%%%%%%%%%%%%%%%%%%%%%%%%%%%%%
\begin{frame}{}
    .
\end{frame}
%%%%%%%%%%%%%%%%%%%%%%%%%%%%%%%%%%%%%%%%%%%%%%%%%%%%%%%%%%%%%%%%%%%%%%%%


%%%%%%%%%%%%%%%%%%%%%%%%%%%%%%%%%%%%%%%%%%%%%%%%%%%%%%%%%%%%%%%%%%%%%%%%
\begin{frame}{}
    .
\end{frame}
%%%%%%%%%%%%%%%%%%%%%%%%%%%%%%%%%%%%%%%%%%%%%%%%%%%%%%%%%%%%%%%%%%%%%%%%

%%%%%%%%%%%%%%%%%%%%%%%%%%%%%%%%%%%%%%%%%%%%%%%%%%%%%%%%%%%%%%%%%%%%%%%%
\begin{frame}{}
    .
\end{frame}
%%%%%%%%%%%%%%%%%%%%%%%%%%%%%%%%%%%%%%%%%%%%%%%%%%%%%%%%%%%%%%%%%%%%%%%%

%%%%%%%%%%%%%%%%%%%%%%%%%%%%%%%%%%%%%%%%%%%%%%%%%%%%%%%%%%%%%%%%%%%%%%%%
\begin{frame}{}
    .
\end{frame}
%%%%%%%%%%%%%%%%%%%%%%%%%%%%%%%%%%%%%%%%%%%%%%%%%%%%%%%%%%%%%%%%%%%%%%%%

%%%%%%%%%%%%%%%%%%%%%%%%%%%%%%%%%%%%%%%%%%%%%%%%%%%%%%%%%%%%%%%%%%%%%%%%
\begin{frame}{}
    .
\end{frame}
%%%%%%%%%%%%%%%%%%%%%%%%%%%%%%%%%%%%%%%%%%%%%%%%%%%%%%%%%%%%%%%%%%%%%%%%

%%%%%%%%%%%%%%%%%%%%%%%%%%%%%%%%%%%%%%%%%%%%%%%%%%%%%%%%%%%%%%%%%%%%%%%%
\begin{frame}{}
    .
\end{frame}
%%%%%%%%%%%%%%%%%%%%%%%%%%%%%%%%%%%%%%%%%%%%%%%%%%%%%%%%%%%%%%%%%%%%%%%%

%%%%%%%%%%%%%%%%%%%%%%%%%%%%%%%%%%%%%%%%%%%%%%%%%%%%%%%%%%%%%%%%%%%%%%%%
\begin{frame}{}
    .
\end{frame}
%%%%%%%%%%%%%%%%%%%%%%%%%%%%%%%%%%%%%%%%%%%%%%%%%%%%%%%%%%%%%%%%%%%%%%%%




\end{document}
