% !TeX spellcheck = en_US
\documentclass[
%handout,
aspectratio=169]{beamer}
\usetheme{default}

\usepackage{tikz}

\usepackage{amsmath,amssymb}

\newcommand{\pitem}{\pause\item}

\newtheorem{statement}{Statement}

\newcommand{\bits}{\{0,1\}}
\newcommand{\bitstr}{\bits^*}
\newcommand{\sshalf}{{\textstyle\frac12}}
\newcommand{\seqn}[2]{{#1}_1,{#1}_2,\dotsc,{#1}_{#2}}
\newcommand{\seqin}[3]{{#1}_{{#2}_1},{#1}_{{#2}_2},\dotsc,{#1}_{{#2}_{#3}}}
\newcommand{\IC}{\mathrm{IC}}
\newcommand{\poly}{\mathrm{poly}}
\newcommand{\Nat}{\mathbb{N}}

\DeclareMathOperator{\dom}{dom}
\DeclareMathOperator{\rank}{rank}
\DeclareMathOperator{\rng}{rng}
\DeclareMathOperator*{\E}{\mathbb{E}}

\DeclareMathOperator{\KL}{D_{KL}}
\DeclareMathOperator{\CE}{CE}
\newcommand{\midd}{\parallel}


\newenvironment{sol}{%
    \begin{block}{}%
        \setbeamercolor{block body}{bg=lightgray}%
    }{
\end{block}}

\AtBeginSection[]{
    \begin{frame}
        \vfill
        \centering
        \begin{beamercolorbox}[sep=8pt,center,shadow=true,rounded=true]{title}
            \usebeamerfont{title}\insertsectionhead\par%
        \end{beamercolorbox}
        \vfill
    \end{frame}
}


\title{Algorithmic approach}
\author{Alexander Smal at NUP}
\date{11.12.2024.\\ Lecture 6}

\begin{document}
%%%%%%%%%%%%%%%%%%%%%%%%%%%%%%%%%%%%%%%%%%%%%%%%%%%%%%%%%%%%%%%%%%%%%%%%
\begin{frame}[plain]
    \maketitle
\end{frame}
%%%%%%%%%%%%%%%%%%%%%%%%%%%%%%%%%%%%%%%%%%%%%%%%%%%%%%%%%%%%%%%%%%%%%%%%

%%%%%%%%%%%%%%%%%%%%%%%%%%%%%%%%%%%%%%%%%%%%%%%%%%%%%%%%%%%%%%%%%%%%%%%%
\begin{frame}{Relative Complexity}
How much information is in the first million digits of $\pi$?

\pause
\begin{definition}
    A partial function $f:\bitstr\to\bitstr$ is \emph{computable} if there exists a program $P$:
    \begin{itemize}
        \item for $\forall x \in \dom f$: $P(x)$ outputs $f(x)$,
        \item for $\forall x \not\in \dom f$: $P(x)$ does not halt.
    \end{itemize}
\end{definition}

\pause
Example: \texttt{unzip} program such that goes into infinite cycle if the input archive is ill-formed.

\pause
\begin{definition}
    Let $F:\bitstr\to\bitstr$ be a computable function. The \emph{complexity of $x$ relative to $F$} is defined as
    \[K_F(x) = \min\{|p| : F(p) = x\}.\]
\end{definition}
\end{frame}
%%%%%%%%%%%%%%%%%%%%%%%%%%%%%%%%%%%%%%%%%%%%%%%%%%%%%%%%%%%%%%%%%%%%%%%%


%%%%%%%%%%%%%%%%%%%%%%%%%%%%%%%%%%%%%%%%%%%%%%%%%%%%%%%%%%%%%%%%%%%%%%%%
\tikzset{
    treenode/.style = {align=center, inner sep=0pt, text centered, font=\sffamily},
    leaf/.style = {treenode, rectangle, draw, color=black!50, text=black, text width=1.5em, minimum height=1em},
    vert/.style = {treenode, circle, draw, text width=1.5em, thick},
    pbox/.style = {treenode, rectangle, draw, text width=2em, minimum height=2em, thick},
    alice/.style = {color=red!60!black!60},
    bob/.style   = {color=blue!60!black!60},
}

%%%%%%%%%%%%%%%%%%%%%%%%%%%%%%%%%%%%%%%%%%%%%%%%%%%%%%%%%%%%%%%%%%%%%%%%
\begin{frame}{Kolmogorov Complexity}
\begin{definition}
    We say that the description language of $F$ is \emph{not worse} than the description language of $G$, and denote it by $F \prec G$, if there exists a constant $c_G$ such that for $\forall x \in \bitstr$,
    \[K_F(x) \le K_G(x) + c_G.\]
\end{definition}

\pause
\begin{theorem}[Solomonov-Kolmogorov]\label{thm:solomonov-kolmogorov}
    There exists a computable function $U$ such that for any other computable function $G$, the relation $U \prec G$ holds.
\end{theorem}

\begin{definition}
    We will call $K(x) = K_U(x)$ the \emph{Kolmogorov complexity of $x$}.
\end{definition}

\end{frame}
%%%%%%%%%%%%%%%%%%%%%%%%%%%%%%%%%%%%%%%%%%%%%%%%%%%%%%%%%%%%%%%%%%%%%%%%


%%%%%%%%%%%%%%%%%%%%%%%%%%%%%%%%%%%%%%%%%%%%%%%%%%%%%%%%%%%%%%%%%%%%%%%%
\begin{frame}[fragile]{Properties of Kolmogorov Complexity}

    \begin{enumerate}
    \item There exists a constant $c$ such that for all $x$, $K(x) \le |x| + c$.
    \item There exists a constant $c$ such that for all $x$, $K(xx) \le |x| + c$.
    \item For any optimal $F_1$ and $F_2$, it holds that $F_1 \prec F_2$ and $F_2 \prec F_1$, i.e., there exists a constant $c$ such that
    \(
    |K_{F_1} - K_{F_2}| \le c.
    \)
\end{enumerate}
\pause
\begin{statement}
    For any $n$, there exists $x \in \{0,1\}^n$ such that $K(x) \ge n$ (i.e., $x$ is incompressible).
\end{statement}

\begin{statement}
    There exists a constant $c > 0$ such that for $99\%$ of words of length $n$:
    \[
    n - c \le K(x) \le n + c = |x| + c.
    \]
\end{statement}
\end{frame}
%%%%%%%%%%%%%%%%%%%%%%%%%%%%%%%%%%%%%%%%%%%%%%%%%%%%%%%%%%%%%%%%%%%%%%%%


%%%%%%%%%%%%%%%%%%%%%%%%%%%%%%%%%%%%%%%%%%%%%%%%%%%%%%%%%%%%%%%%%%%%%%%%
\begin{frame}{Incomputability of Kolmogorov Complexity}
    \begin{statement}
        There does not exist a computable function $f:\bitstr\to\bitstr$ that is
        defined everywhere and satisfies $f(\bar n) = x_n$, where $K(x_n) \ge n$ ($\bar n$ denotes the binary representation of the number $n$).
    \end{statement}
\pause
\begin{corollary}
    The mapping $x \to K(x)$ is not computable.
\end{corollary}
\pause
    This is similar to Berry's paradox.
\begin{center}
    The smallest positive integer not definable in under sixty letters.
\end{center}
This phrase contains 57 letters and defines that very smallest number.
\pause
\begin{corollary}
The optimal description language corresponds to a function that is undefined on infinite number of inputs.
\end{corollary}
\end{frame}
%%%%%%%%%%%%%%%%%%%%%%%%%%%%%%%%%%%%%%%%%%%%%%%%%%%%%%%%%%%%%%%%%%%%%%%%

%%%%%%%%%%%%%%%%%%%%%%%%%%%%%%%%%%%%%%%%%%%%%%%%%%%%%%%%%%%%%%%%%%%%%%%%
\begin{frame}{Incomputability of Kolmogorov Complexity: corollaries}
\begin{corollary}[Chaitin's theorem]
    Let there be some formal theory such that we can write `$K(x) > c$'. For all sufficiently large $c$ and for all $x$, the formulas `$K(x) > c$' are unprovable (and almost all of these statements are true).
\end{corollary}

\begin{corollary}[Gödel's first incompleteness theorem]
    Any consistent formal system $F$ within which a certain amount of elementary arithmetic can be carried out is incomplete; i.e. there are statements of the language of $F$ which can neither be proved nor disproved in $F$.
\end{corollary}

This, among other things, provides a way to generate unprovable statements with good probability.

\end{frame}
%%%%%%%%%%%%%%%%%%%%%%%%%%%%%%%%%%%%%%%%%%%%%%%%%%%%%%%%%%%%%%%%%%%%%%%%

%%%%%%%%%%%%%%%%%%%%%%%%%%%%%%%%%%%%%%%%%%%%%%%%%%%%%%%%%%%%%%%%%%%%%%%%
\begin{frame}{Connection to Shannon Entropy}
\begin{statement}\label{st:kologorov:entropy}
    Let $x = \langle{011010010\dotso 10110}\rangle$ of length $n$ contain $p \cdot n$ ones and $(1-p) \cdot n$ zeros, then
    \[
    K(x) \le \left(p \cdot \log \frac{1}{p} + (1-p) \cdot \log \frac{1}{1-p}\right) \cdot n + O(\log n).
    \]
    \end{statement}
\end{frame}
%%%%%%%%%%%%%%%%%%%%%%%%%%%%%%%%%%%%%%%%%%%%%%%%%%%%%%%%%%%%%%%%%%%%%%%%

%%%%%%%%%%%%%%%%%%%%%%%%%%%%%%%%%%%%%%%%%%%%%%%%%%%%%%%%%%%%%%%%%%%%%%%%
\begin{frame}{Connection to Shannon Entropy: proof}
    \begin{proof}
    Consider the following description:
    \begin{center}
        $\langle$number of '1's, number of '0's, index of the permutation with the given number of '1's and '0's$\rangle$.
    \end{center}
    The total number of permutations is
    \[
    \binom{n}{pn} = 2^{\left(p \cdot \log \frac{1}{p} + (1-p) \cdot \log \frac{1}{1-p}\right) \cdot n + O(\log n)}.
    \]
    Thus, $K(x) \le \left(p \cdot \log \frac{1}{p} + (1-p) \cdot \log \frac{1}{1-p}\right) \cdot n + O(\log n) = H(p) \cdot n + O(\log n)$.
    \end{proof}
    \pause
    Note: in the proof, it is important to encode this triplet in such a way that it can be uniquely split into three parts. For example, one could double all the bits of the first components and add a separator '01'.
\end{frame}
%%%%%%%%%%%%%%%%%%%%%%%%%%%%%%%%%%%%%%%%%%%%%%%%%%%%%%%%%%%%%%%%%%%%%%%%

%%%%%%%%%%%%%%%%%%%%%%%%%%%%%%%%%%%%%%%%%%%%%%%%%%%%%%%%%%%%%%%%%%%%%%%%
\begin{frame}[fragile]{Conditional Kolmogorov Complexity}
\begin{definition} The complexity of the \emph{conditional description} of $x$ given $y$ relative to $F$ is:
    \[K_F(x \mid y) = \min\{|p| : F(p,y) = x\}.\]
A conditional description $F$ is \emph{not worse} than a conditional description $G$, denoted $F \prec G$, if there exists a constant $c$ such that for any $x$ and $y$,
    \[
    K_F(x \mid y) \le K_G(x \mid y) + c.
    \]
\end{definition}\vspace{-5mm}
\pause
\begin{theorem}
    There exists an optimal method for conditional descriptions $U$ such that for any other method of conditional description $G$, it holds that $U \prec G$.
\end{theorem}
\pause
\begin{definition}
    The complexity of $x$ given $y$ relative to the optimal method of conditional description is called the \emph{conditional Kolmogorov complexity of $x$ given $y$}, denoted $K(x \mid y)$.
\end{definition}

\end{frame}
%%%%%%%%%%%%%%%%%%%%%%%%%%%%%%%%%%%%%%%%%%%%%%%%%%%%%%%%%%%%%%%%%%%%%%%%

%%%%%%%%%%%%%%%%%%%%%%%%%%%%%%%%%%%%%%%%%%%%%%%%%%%%%%%%%%%%%%%%%%%%%%%%
\begin{frame}{Properties of Conditional Kolmogorov Complexity}
    \begin{enumerate}
    \item $K(x \mid y) \le K(x) + O(1)$.
    \item $K(x \mid y) \le |x| + O(1)$.
    \item There exists a constant $c$ such that for all $n$, for all $y$, for $99\%$ of words $x$ of length $n$, the following holds: \(|K(x \mid y) - n| \le c\).
    \item $K(x \mid x) = O(1)$.
    \item Let $f$ be a computable function. Then there exists a constant $c_f$ such that for all $x$, $K(f(x) \mid x) \le c_f$.
\end{enumerate}
\end{frame}
%%%%%%%%%%%%%%%%%%%%%%%%%%%%%%%%%%%%%%%%%%%%%%%%%%%%%%%%%%%%%%%%%%%%%%%%

%%%%%%%%%%%%%%%%%%%%%%%%%%%%%%%%%%%%%%%%%%%%%%%%%%%%%%%%%%%%%%%%%%%%%%%%
\begin{frame}{Complexity of a Pair}
We denote the complexity of a pair by $K(x,y) = K(\langle x,y\rangle)$, where $\langle\cdot,\cdot\rangle$ is an arbitrary computable encoding of pairs.

\begin{statement}
    The following statement is \uncover<2->{\alert{false}:}
    \[
    \exists c\ \forall x,y,\ K(x,y) \le K(x) + K(y \mid x) + c.
    \]
\end{statement}\vspace{-5mm}

\pause\pause

\begin{theorem}
    \[
    \forall x,y,\ K(x,y) \le K(x) + K(y \mid x) + O(\log K(x,y)).
    \]
\end{theorem}

\end{frame}
%%%%%%%%%%%%%%%%%%%%%%%%%%%%%%%%%%%%%%%%%%%%%%%%%%%%%%%%%%%%%%%%%%%%%%%%


%%%%%%%%%%%%%%%%%%%%%%%%%%%%%%%%%%%%%%%%%%%%%%%%%%%%%%%%%%%%%%%%%%%%%%%%
\begin{frame}{Kolmogorov-Levin Theorem}
\begin{theorem}[Kolmogorov-Levin]\label{thm:kolmogorov-levin}
    $|K(x,y) - K(x) - K(y \mid x)| \le O(\log K(x,y))$.
\end{theorem}

\begin{definition}
    \emph{Mutual information of $x$ and $y$:}
    \[
    \begin{aligned}
        I(x:y) &= K(y) - K(y \mid x),\\
        I(y:x) &= K(x) - K(x \mid y).
    \end{aligned}
    \]
\end{definition}

The Kolmogorov-Levin theorem states the symmetry of mutual information.
\[
I(x:y) = K(x) + K(y) - K(x,y) + O(\log K(x,y)) = I(y:x).
\]
\vspace{-5mm}
\begin{corollary}
    $|I(x:y) - I(y:x)| \le O(\log K(x,y))$.
\end{corollary}
\end{frame}
%%%%%%%%%%%%%%%%%%%%%%%%%%%%%%%%%%%%%%%%%%%%%%%%%%%%%%%%%%%%%%%%%%%%%%%%


%%%%%%%%%%%%%%%%%%%%%%%%%%%%%%%%%%%%%%%%%%%%%%%%%%%%%%%%%%%%%%%%%%%%%%%%
\begin{frame}{Kolmogorov-Levin Theorem: proof}
    The inequality `$\le$' has already been proven. It remains to prove `$\ge$'.
    \[
    \underbrace{K(x)}_{m} + \underbrace{K(y \mid x)}_l \le \underbrace{K(x,y)}_n + \underbrace{O(\log K(x,y))}_{O(\log n)}.
    \]
    Let $S = \{(a,b) \mid K(a,b) \le n\}$. Note that $(x,y) \in S$ and $|S| \le 2^{n+1}$.\\
    Consider $S_x = \{(x,b) \mid (x,b) \in S\}$.
    By definition, $(x,y) \in S_x$.
    \\We will show that
    \[
    l = K(y \mid x) \le \log |S_x| + O(\log n).
    \]
    We will enumerate the set $S$. During this enumeration, we will obtain points from $S_x$. To specify $y$, we need to indicate the index of $(x,y)$ in this enumeration. Additionally, to start such an enumeration, we need to know the number $n$. Thus,
    \[|S_x| \ge 2^{l - c \cdot \log n} \ge 2^{l'},\]
    where $l'$ is the nearest integer less than or equal to $l$, i.e., $l' = \lfloor l - c \cdot \log n \rfloor$.

\end{frame}
%%%%%%%%%%%%%%%%%%%%%%%%%%%%%%%%%%%%%%%%%%%%%%%%%%%%%%%%%%%%%%%%%%%%%%%%

%%%%%%%%%%%%%%%%%%%%%%%%%%%%%%%%%%%%%%%%%%%%%%%%%%%%%%%%%%%%%%%%%%%%%%%%
\begin{frame}{Kolmogorov-Levin Theorem: proof (cont.)}
    Let us look again at the enumeration of $S$. During the enumeration, we encounter ``heavy sections''—those in which the number of elements is at least $2^{l'}$. To specify the section $S_x$, we need to indicate its ordinal number in the enumeration of $S$ among all ``heavy sections''. Thus,
    \[
    m = K(x) \le \log(\text{\# heavy sections}) + O(\log n) + O(\log l').
    \]
    The number of heavy sections is at most $|S|/2^{l'}$.
    \[
    m = K(x) \le \log \frac{|S|}{2^{l'}} + O(\log n) = n - l + O(\log n).
    \]
    Thus, we obtain the statement of the theorem: $m + l \le n + O(\log n)$.

\end{frame}
%%%%%%%%%%%%%%%%%%%%%%%%%%%%%%%%%%%%%%%%%%%%%%%%%%%%%%%%%%%%%%%%%%%%%%%%

%%%%%%%%%%%%%%%%%%%%%%%%%%%%%%%%%%%%%%%%%%%%%%%%%%%%%%%%%%%%%%%%%%%%%%%%
\begin{frame}{Kolmogorov-Levin Theorem: final remark}
    Choose $n$ such that its binary representation is incompressible, i.e., $K(\bar n) = \log n + O(1)$. Take $x \in \{0,1\}^n$ such that $K(x \mid \bar n) = n + O(1)$. Then
\begin{itemize}
    \item\(
    I(\bar n : x) = K(x) - K(x \mid \bar n) = n + O(1) - (n + O(1)) = O(1),
    \)
    \item\(
    I(x : \bar n) = K(\bar n) - K(\bar n \mid x) = (\log n + O(1)) - O(1) = \log n + O(1).
    \)
\end{itemize}
Thus, it is not possible to reduce the logarithmic gap in the Kolmogorov-Levin theorem.
\end{frame}

\end{document}
