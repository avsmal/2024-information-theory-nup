% !TeX spellcheck = en_US
\documentclass[
%handout,
aspectratio=169]{beamer}
\usetheme{default}

\usepackage{tikz}

\usepackage{amsmath,amssymb}

\newcommand{\pitem}{\pause\item}

\newtheorem{statement}{Statement}

\newcommand{\bits}{\{0,1\}}
\newcommand{\bitstr}{\bits^*}
\newcommand{\sshalf}{{\textstyle\frac12}}
\newcommand{\seqn}[2]{{#1}_1,{#1}_2,\dotsc,{#1}_{#2}}
\newcommand{\seqin}[3]{{#1}_{{#2}_1},{#1}_{{#2}_2},\dotsc,{#1}_{{#2}_{#3}}}
\newcommand{\IC}{\mathrm{IC}}
\newcommand{\poly}{\mathrm{poly}}
\newcommand{\Nat}{\mathbb{N}}

\DeclareMathOperator{\dom}{dom}
\DeclareMathOperator{\rank}{rank}
\DeclareMathOperator{\rng}{rng}
\DeclareMathOperator*{\E}{\mathbb{E}}

\newenvironment{sol}{%
    \begin{block}{}%
        \setbeamercolor{block body}{bg=lightgray}%
    }{
\end{block}}

\AtBeginSection[]{
    \begin{frame}
        \vfill
        \centering
        \begin{beamercolorbox}[sep=8pt,center,shadow=true,rounded=true]{title}
            \usebeamerfont{title}\insertsectionhead\par%
        \end{beamercolorbox}
        \vfill
    \end{frame}
}


\title{Communication Complexity}
\author{Alexander Smal at NUP}
\date{20.11.2024.\\ Lecture 6}

\begin{document}
%%%%%%%%%%%%%%%%%%%%%%%%%%%%%%%%%%%%%%%%%%%%%%%%%%%%%%%%%%%%%%%%%%%%%%%%
\begin{frame}[plain]
    \maketitle
\end{frame}
%%%%%%%%%%%%%%%%%%%%%%%%%%%%%%%%%%%%%%%%%%%%%%%%%%%%%%%%%%%%%%%%%%%%%%%%

%%%%%%%%%%%%%%%%%%%%%%%%%%%%%%%%%%%%%%%%%%%%%%%%%%%%%%%%%%%%%%%%%%%%%%%%
\begin{frame}{Communication Complexity Model}
    \begin{itemize}
    \item Let $X$, $Y$, and $Z$ be finite sets, and let $f: X \times Y \to Z$.

    \pitem Two players, Alice and Bob, solve the \emph{communication problem for the function $f$}:
    \begin{enumerate}
        \item The sets $X$, $Y$, $Z$, and the function $f$ are known to both players,
        \item Alice knows some $x \in X$,
        \item Bob knows some $y \in Y$,
        \item Alice and Bob want to compute $f(x,y)$.
    \end{enumerate}

    \pitem To solve the communication problem, Alice and Bob can exchange bit messages.

    \pitem The problem is solved if both players know $f(x,y)$.

    \pitem We are interested in the minimum number of bits that need to be sent in order to compute $f(x,y)$.
    \end{itemize}

\end{frame}
%%%%%%%%%%%%%%%%%%%%%%%%%%%%%%%%%%%%%%%%%%%%%%%%%%%%%%%%%%%%%%%%%%%%%%%%


%%%%%%%%%%%%%%%%%%%%%%%%%%%%%%%%%%%%%%%%%%%%%%%%%%%%%%%%%%%%%%%%%%%%%%%%
\tikzset{
    treenode/.style = {align=center, inner sep=0pt, text centered, font=\sffamily},
    leaf/.style = {treenode, rectangle, draw, color=black!50, text=black, text width=1.5em, minimum height=1em},
    vert/.style = {treenode, circle, draw, text width=1.5em, thick},
    pbox/.style = {treenode, rectangle, draw, text width=2em, minimum height=2em, thick},
    alice/.style = {color=red!60!black!60},
    bob/.style   = {color=blue!60!black!60},
}

%%%%%%%%%%%%%%%%%%%%%%%%%%%%%%%%%%%%%%%%%%%%%%%%%%%%%%%%%%%%%%%%%%%%%%%%
\begin{frame}{Communication Complexity Model}
    Introduced by Andrew Yao in 1979.\medskip
    \begin{center}
        \begin{tikzpicture}
            \node[label=above:Alice]    (A) at (0,3.5)
            {\includegraphics[height=1cm]{alice}};
            \node[label=above:Bob]      (B) at (8,3.5)
            {\includegraphics[height=1cm]{bob}};
            \node () at (4,4) {};
            \node () at (4,-2.1) {};

            \node () at (0,2) {\phantom{$x\in\{0,1\}^n$}};
            \node () at (8,2) {\phantom{$y\in\{0,1\}^n$}};

            \node<2-> () at (0,2) {$x\in\{0,1\}^n$};
            \node<2-> () at (8,2) {$y\in\{0,1\}^n$};

            \path<4-> (1.5,4.5) edge[->,thick] node[below]{$b_1$} (6.5,4.5);
            \path<5-> (1.5,3.5) edge[->,thick] node[below]{$b_2$} (6.5,3.5);
            \path<6-> (1.5,2.5) edge[<-,thick] node[below]{$b_3$} (6.5,2.5);

            \node<3-> () at (4,0) {Alice and Bob want to compute $f(x,y)$.};

        \end{tikzpicture}
    \end{center}
\end{frame}
%%%%%%%%%%%%%%%%%%%%%%%%%%%%%%%%%%%%%%%%%%%%%%%%%%%%%%%%%%%%%%%%%%%%%%%%


%%%%%%%%%%%%%%%%%%%%%%%%%%%%%%%%%%%%%%%%%%%%%%%%%%%%%%%%%%%%%%%%%%%%%%%%
\begin{frame}[fragile]{Communication Protocol}
    \begin{center}
        \begin{tikzpicture}[->,level/.style={sibling distance = 3cm/#1,
                level distance = 1.1cm}]
            \node [vert,alice] {A}
            child{ node [vert, alice] {A}
                child{ node [vert, bob] {B}
                    child{ node [leaf] {0} edge from parent node[above left] {$0$}} %for a named pointer
                    child{ node [vert, bob] {B}
                        child{ node [leaf] {0} edge from parent node[above left]{$0$}}
                        child{ node [leaf] {1} edge from parent node[above right]{$1$}}
                        edge from parent node[above right]{$1$}
                    }
                    edge from parent node[above left]{$0$}
                }
                child{ node [leaf] {1} edge from parent node[above right]{$1$}}
                edge from parent node[above left]{$0$}
            }
            child{ node [vert, bob] {B}
                child{ node [vert, alice] {A}
                    child{ node [vert, bob]  {B}
                        child{ node [leaf] {1} edge from parent node[above left]{$0$}}
                        child{ node [leaf] {0} edge from parent node[above right]{$1$}}
                        edge from parent node[above left]{$0$}
                    }
                    child{ node [leaf] {1} edge from parent node[above right]{$1$}}
                    edge from parent node[above left]{$0$}
                }
                child{node [leaf] {0} edge from parent node[above right]{$1$}}
                edge from parent node[above right]{$1$}
            }
            ;
            \node[label=above:Alice]    (A) at (-4,-1) {\includegraphics[height=1cm]{alice}};
            \node[label=above:Bob]      (B) at (4, -1) {\includegraphics[height=1cm]{bob}};
        \end{tikzpicture}
    \end{center}
    \emph{Communication complexity} of $f$ is a minimal depth of
    a~protocol solving $f$, denoted $D(f)$. The same definition for relations.
\end{frame}
%%%%%%%%%%%%%%%%%%%%%%%%%%%%%%%%%%%%%%%%%%%%%%%%%%%%%%%%%%%%%%%%%%%%%%%%


%%%%%%%%%%%%%%%%%%%%%%%%%%%%%%%%%%%%%%%%%%%%%%%%%%%%%%%%%%%%%%%%%%%%%%%%
\begin{frame}{Examples}
    \begin{statement}
        For any $f: \{0,1\}^n \times\{0,1\}^n \to \bits$, $D(f) \le n + 1$.
    \end{statement}

    \pause
    Examples of functions with non-trivial upper bounds on communication complexity.
    \begin{enumerate}
        \item (Pointer Chasing) $D(\mathrm{PC}) \le k \log n$,
        where \[PC(x,y) = \underbrace{x(y(x(y\dotsb(x(y(x(y(x}_{\text{$k$ rounds}}(0))))))\dotsb))),\]
        $x$ and $y$ define a bipartite directed graph on $2n$ vertices with out-degree each vertex equal to $1$.
        They start with Alice's vertex $0$ and output the final vertex.


        \pitem $D(\mathrm{MED}) = O(\log^2 n)$, where $x$ and $y$ are interpreted as characteristic functions of subsets of $[n]$, and $\mathrm{MED}(x,y)$ is the median of their union.

        \pitem $D(\mathrm{CIS}_G) = O(\log^2 n)$, where $x$ is interpreted as the characteristic function of some clique in the graph $G$, and $y$ as the characteristic function of some independent set in the graph $G$. $\mathrm{CIS}(x,y) = 1$ if the clique and the independent set share a vertex.
    \end{enumerate}

\end{frame}
%%%%%%%%%%%%%%%%%%%%%%%%%%%%%%%%%%%%%%%%%%%%%%%%%%%%%%%%%%%%%%%%%%%%%%%%

%%%%%%%%%%%%%%%%%%%%%%%%%%%%%%%%%%%%%%%%%%%%%%%%%%%%%%%%%%%%%%%%%%%%%%%%
\begin{frame}{Lower Bounds}
    Consider a communication protocol for $f: X \times Y \to Z$. For each vertex $v$, we define the set $R_v \subset X \times Y$—the set of all pairs $(x,y) \in X \times Y$ for which the computation reaches vertex $v$.

    \pause
    \begin{statement}
        For all vertices $v$, the set $R_v$ is a combinatorial rectangle, i.e., there exist subsets $X_v \subset X$ and $Y_v \subset Y$ such that $R_v = X_v \times Y_v$.
    \end{statement}

    \pause
    \begin{corollary}
        The leaves of the communication protocol for the function $f$ define a partition of the set $X \times Y$ into monochromatic rectangles.
    \end{corollary}

    Let $C^R(f)$ denote the minimum number of \emph{monochromatic} rectangles covering $X \times Y$.

    \pause
    \begin{statement}
        $D(f) \ge \log C^R(f)$.
    \end{statement}

\end{frame}
%%%%%%%%%%%%%%%%%%%%%%%%%%%%%%%%%%%%%%%%%%%%%%%%%%%%%%%%%%%%%%%%%%%%%%%%

%%%%%%%%%%%%%%%%%%%%%%%%%%%%%%%%%%%%%%%%%%%%%%%%%%%%%%%%%%%%%%%%%%%%%%%%
\begin{frame}{Rectangle Size Method}
    We define a certain semi-additive measure on subrectangles of $X \times Y$. Then the following estimate holds:
    \[
    C^R(f) \ge \frac{w(X \times Y)}{\max\limits_{\text{mono. } A \times B} w(A \times B)}.
    \]

    \pause
    \begin{block}{Fooling Set Method}
        This is a special case of the rectangle size method, where a certain set $F \subset X \times Y$ is fixed, and the measure $w$ is defined as follows:
        \[
        w(A \times B) = |(A \times B) \cap F|.
        \]
        If no monochromatic rectangle contains more than one element from $F$, then $C^R(f) \ge |F|$.
    \end{block}
\end{frame}
%%%%%%%%%%%%%%%%%%%%%%%%%%%%%%%%%%%%%%%%%%%%%%%%%%%%%%%%%%%%%%%%%%%%%%%%

%%%%%%%%%%%%%%%%%%%%%%%%%%%%%%%%%%%%%%%%%%%%%%%%%%%%%%%%%%%%%%%%%%%%%%%%
\begin{frame}{Matrix Rank Method}
    \begin{itemize}
        \item Consider the \emph{function matrix $f$}---a matrix where
        \begin{itemize}
            \item the rows are indexed by the elements of $X$,
            \item the columns by the elements of $Y$,
            \item the entry $(x,y)$ contains $f(x,y)$.
        \end{itemize}
        \pitem Consider the function matrix as a matrix $M$ over some sufficiently large field.
        \pitem One can show that $C^R(f) \ge \rank M$.
    \end{itemize}
\end{frame}
%%%%%%%%%%%%%%%%%%%%%%%%%%%%%%%%%%%%%%%%%%%%%%%%%%%%%%%%%%%%%%%%%%%%%%%%

%%%%%%%%%%%%%%%%%%%%%%%%%%%%%%%%%%%%%%%%%%%%%%%%%%%%%%%%%%%%%%%%%%%%%%%%
\begin{frame}{Exercises}

\begin{statement}
    $D(\mathrm{EQ}_n) = n + 1$, where $\mathrm{EQ}_n(x,y) = 1 \iff x = y$.
\end{statement}

\pause
\begin{statement}
    $D(\mathrm{GE}_n) = n + 1$, where $\mathrm{GE}_n(x,y) = 1 \iff x \ge y$.
\end{statement}

\end{frame}
%%%%%%%%%%%%%%%%%%%%%%%%%%%%%%%%%%%%%%%%%%%%%%%%%%%%%%%%%%%%%%%%%%%%%%%%

%%%%%%%%%%%%%%%%%%%%%%%%%%%%%%%%%%%%%%%%%%%%%%%%%%%%%%%%%%%%%%%%%%%%%%%%
\begin{frame}{Probabilistic Protocols}
    \begin{itemize}
    \item Alice receives a pair $(x,r)$ as input, where $x \in X$, and $r$ is a random string.
    \pitem Bob receives a pair $(y,s)$ where $y \in Y$, and $s$ is a random string.
    \pitem The functions $g_v$ and $h_v$ written at the vertices of the protocol for such a game will take two arguments—the input and the random string, i.e., the messages sent can depend on the random bits.

    \pitem Accordingly, the result of the game will depend on $x,y,r,s$.
    \end{itemize}

    \pause
    \begin{definition}
        A probabilistic protocol $\epsilon$-computes $f$ if for any pair $x,y$, with probability (over the choice of $(r,s)$) at least $1 - \epsilon$, the result of the protocol equals $f(x,y)$. The minimum height of a probabilistic protocol that $\epsilon$-computes $f$ is denoted by $R^\epsilon(f)$.
    \end{definition}

\end{frame}
%%%%%%%%%%%%%%%%%%%%%%%%%%%%%%%%%%%%%%%%%%%%%%%%%%%%%%%%%%%%%%%%%%%%%%%%

%%%%%%%%%%%%%%%%%%%%%%%%%%%%%%%%%%%%%%%%%%%%%%%%%%%%%%%%%%%%%%%%%%%%%%%%
\begin{frame}{Examples}
    \begin{statement}
        $R^\epsilon(\mathrm{EQ}_n) = O(\log n + \log(1/\epsilon))$.
    \end{statement}

    \pause
    \begin{statement}
        $R^\epsilon(\mathrm{GE}_n) = O(\log n(\log n + \log(1/\epsilon)))$.
    \end{statement}

    \pause
    \begin{statement}
        If Alice and Bob have access to a common source of randomness (i.e., $r = s$), they can $\epsilon$-compute the predicate $\mathrm{EQ}_n$ by sending $O(\log(1/\epsilon))$ bits.
    \end{statement}

    \pause
    \begin{statement}
        If Alice and Bob have access to a common source of randomness, then for any fixed positive $\epsilon$, they can $\epsilon$-compute the predicate $\mathrm{GE}_n$ with an error of at most $\epsilon$ by sending $O(\log n)$ bits.
    \end{statement}
\end{frame}
%%%%%%%%%%%%%%%%%%%%%%%%%%%%%%%%%%%%%%%%%%%%%%%%%%%%%%%%%%%%%%%%%%%%%%%%


\section{Connection Between Protocols and Formulas}
%%%%%%%%%%%%%%%%%%%%%%%%%%%%%%%%%%%%%%%%%%%%%%%%%%%%%%%%
\begin{frame}[fragile]{Karchmer--Wigderson game}
    The \alert{Karchmer--Wigderson game} for $f:\{0,1\}^n\to \{0,1\}$:
    \begin{center}
        \begin{tikzpicture}[->,level/.style={sibling distance = 3cm/#1,
                level distance = 1.2cm}]
            \node [vert,alice] {A}
            child{ node [vert, alice] {A}
                child{ node [vert, bob] {B}
                    child{ node [leaf] {1} edge from parent node[above left] {$0$}} %for a named pointer
                    child{ node [vert, bob] {B}
                        child{ node [leaf] {2} edge from parent node[above left]{$0$}}
                        child{ node [leaf] {3} edge from parent node[above right]{$1$}}
                        edge from parent node[above right]{$1$}
                    }
                    edge from parent node[above left]{$0$}
                }
                child{ node [leaf] {1} edge from parent node[above right]{$1$}}
                edge from parent node[above left]{$0$}
            }
            child{ node [vert, bob] {B}
                child{ node [vert, alice] {A}
                    child{ node [vert, bob]  {B}
                        child{ node [leaf] {4} edge from parent node[above left]{$0$}}
                        child{ node [leaf] {1} edge from parent node[above right]{$1$}}
                        edge from parent node[above left]{$0$}
                    }
                    child{ node [leaf] {2} edge from parent node[above right]{$1$}}
                    edge from parent node[above left]{$0$}
                }
                child{node [leaf] {3} edge from parent node[above right]{$1$}}
                edge from parent node[above right]{$1$}
            }
            ;
            \node[label=above:Alice,label=below:$x\in f^{-1}(0)$]
            (A) at (-4,-1) {\includegraphics[height=1cm]{alice}};
            \node[label=above:Bob, label=below:$y\in f^{-1}(1)$]
            (B) at (4, -1) {\includegraphics[height=1cm]{bob}};
        \end{tikzpicture}
    \end{center}
    Players want to find an index $i\in[n]$ such that $x_i\neq
    y_i$.\medskip

    \pause Communication game for $R_f = \{((x,y),i) \mid x \in f^{-1}(0), y \in f^{-1}(1), x_i \neq y_i\}.$

\end{frame}
%%%%%%%%%%%%%%%%%%%%%%%%%%%%%%%%%%%%%%%%%%%%%%%%%%%%%%%%

%%%%%%%%%%%%%%%%%%%%%%%%%%%%%%%%%%%%%%%%%%%%%%%%%%%%%%%%
\begin{frame}[fragile]{Karchmer-Wigderson Theorem}
    \centering
        \begin{tikzpicture}[->,level/.style={sibling distance = 2.5cm/#1,
                level distance = 1.2cm}]
            \begin{scope}
            \node [vert,alice] {A}
            child{ node [vert, alice] {A}
                child{ node [vert, bob] {B}
                    child{ node [leaf] {1} } %for a named pointer
                    child{ node [vert, bob] {B}
                        child{ node [leaf] {2} }
                        child{ node [leaf] {3} }
                    }
                }
                child{ node [leaf] {1} }
            }
            child{ node [vert, bob] {B}
                child{ node [vert, alice] {A}
                    child{ node [vert, bob]  {B}
                        child{ node [leaf] {4} }
                        child{ node [leaf] {1} }
                    }
                    child{ node [leaf] {2} }
                }
                child{node [leaf] {3}}
            }
            ;
            \node at (3.5, -2.5) {$\Longleftrightarrow$};
            \node at(0, -5.5) {Protocol for $KW_f$};
        \end{scope}
        \begin{scope}[xshift=7cm]
            \node [vert,alice] {$\land$}
            child{ node [vert, alice] {$\land$}
                child{ node [vert, bob] {$\lor$}
                    child{ node [leaf] {$x_1$}} %for a named pointer
                    child{ node [vert, bob] {$\lor$}
                        child{ node [leaf] {$\lnot x_2$}}
                        child{ node [leaf] {$x_3$}}
                    }
                }
                child{ node [leaf] {$x_1$}}
            }
            child{ node [vert, bob] {$\lor$}
                child{ node [vert, alice] {$\land$}
                    child{ node [vert, bob]  {$\lor$}
                        child{ node [leaf] {$\lnot x_4$}}
                        child{ node [leaf] {$x_1$}}
                    }
                    child{ node [leaf] {$x_2$}}
                }
                child{node [leaf] {$\lnot x_3$}}
            }
            ;
            \node at(0, -5.5) {De~Morgan formula for $f$};
        \end{scope}
        \end{tikzpicture}

        \pause
        \textbf{Corollary:} For any function $f$, $L(f) = L(R_f)$.
\end{frame}
%%%%%%%%%%%%%%%%%%%%%%%%%%%%%%%%%%%%%%%%%%%%%%%%%%%%%%%%

%%%%%%%%%%%%%%%%%%%%%%%%%%%%%%%%%%%%%%%%%%%%%%%%%%%%%%%%%%%%%%%%%%%%%%%%
\begin{frame}{Theorems}
   \begin{theorem}[Shannon]
       There exists a function $f$ such that $L(f) = \Omega(2^n/n)$.
   \end{theorem}

   \pause
   \begin{theorem}[Håstad]
        There exists an \emph{implicit} function $f$ such that $L(f) = \tilde\Omega(n^3)$.
   \end{theorem}

   \pause
   \begin{theorem}[Bonet, Buss]
        For any $\alpha > 1$, such that for any formula $\phi$ of size $s$, there exists an equivalent formula $\phi'$ of size $s^\alpha$ and depth $O(\log s)$ (the constant depends on $\alpha$).
   \end{theorem}
\end{frame}
%%%%%%%%%%%%%%%%%%%%%%%%%%%%%%%%%%%%%%%%%%%%%%%%%%%%%%%%%%%%%%%%%%%%%%%%

%%%%%%%%%%%%%%%%%%%%%%%%%%%%%%%%%%%%%%%%%%%%%%%%%%%%%%%%%%%%%%%%%%%%%%%%
\begin{frame}{Information Cost}

    \begin{itemize}
        \item
            Let $\mu$ be some distribution over the inputs of Alice and Bob,
            and $X$, $Y$ be the corresponding random variables.

        \pitem The \emph{external information cost} of protocol $\Pi$ on distribution $\mu$ is:
            $$\IC_\mu^{ext}(\Pi) = I(\Pi(X,Y) : X, Y).$$


        \pitem The \emph{internal information cost} of protocol $\Pi$ on distribution $\mu$ is:
            $$\IC_\mu^{int}(\Pi) = I(\Pi(X,Y) : X \mid Y) + I(\Pi(X,Y) : Y \mid X).$$
    \end{itemize}

    \pause
    \begin{lemma}
        For any protocol $\Pi$ and any distribution $\mu$,
        \[
        D(\Pi) \ge \IC_\mu^{ext}(\Pi) \ge \IC_\mu^{int}(\Pi).
        \]
    \end{lemma}

\end{frame}
%%%%%%%%%%%%%%%%%%%%%%%%%%%%%%%%%%%%%%%%%%%%%%%%%%%%%%%%%%%%%%%%%%%%%%%%

%%%%%%%%%%%%%%%%%%%%%%%%%%%%%%%%%%%%%%%%%%%%%%%%%%%%%%%%%%%%%%%%%%%%%%%%
\begin{frame}{Hardest distribution}
\begin{theorem}
    Let $\Pi$ be a communication protocol. For any distribution $\mu$:
    \(
    \log L(\Pi) \ge \IC_\mu^{ext}(\Pi).
    \)
    Furthermore, there exists a distribution $\mu^*$ for which
    \(
    \log L(\Pi) = \IC_{\mu^*}^{ext}(\Pi).
    \)
    We will call $\mu^*$ the \emph{hardest} distribution for $\Pi$.
\end{theorem}\pause
\begin{proof} For deterministic protocols, $\IC^{ext}(\Pi) = H_\mu(\Pi)$.
    The first statement of the theorem follows from the upper bound on entropy (the entropy
    of a random variable does not exceed the logarithm of the number of outcomes):
    \[
    \IC_\mu^{ext}(\Pi) = H_\mu(\Pi) \le \log L(\Pi).
    \]

    \pause
    To prove the second statement, we present the distribution $\mu^*$:
    we randomly (uniformly) select a leaf $l$ of the protocol $\Pi$ and in the corresponding rectangle $R_l$, we choose any pair $(x,y)$. The resulting distribution $\mu^*$ is uniform over the leaves of $\Pi$,
    \[
    \IC_{\mu^*}^{ext}(\Pi) = H_{\mu^*}(\Pi) = \log L(\Pi).
    \]
\end{proof}
\end{frame}
%%%%%%%%%%%%%%%%%%%%%%%%%%%%%%%%%%%%%%%%%%%%%%%%%%%%%%%%%%%%%%%%%%%%%%%%

%%%%%%%%%%%%%%%%%%%%%%%%%%%%%%%%%%%%%%%%%%%%%%%%%%%%%%%%%%%%%%%%%%%%%%%%
\begin{frame}{Khrapchenko's Theorem}
\begin{theorem}[Khrapchenko]
    $L(\oplus_n) \ge n^2$.
\end{theorem}
\begin{proof}
    We will show that for any protocol there exists a distribution $\mu$: $\IC^{ext}_\mu(\Pi) \ge 2\log n$.
    \pause
    It immediately implies that $L(\oplus_n) \ge n^2$.
    The distribution $\mu$ will be the uniform distribution over pairs of the form $(x,x \oplus e_i)$,
    where $\oplus_n(x) = 0$, and the string $e_i$ has a 1 in position $i$ and 0s in all other positions.
    \pause
    \[
    \IC_\mu^{ext}(\Pi)
    \ge \IC_\mu^{int}(\Pi)
    = I(\Pi:X \mid Y) + I(\Pi:Y \mid X).
    \]
    \pause Let's consider one of the terms $I(\Pi:X \mid Y)$.
    \[\begin{aligned}
        I(\Pi:X \mid Y) &= H(X \mid Y) - H(X \mid Y,\Pi)\\
        &= H(i \mid Y) - H(i \mid Y,\Pi) = \log n - 0.
    \end{aligned}\]
    \pause Thus, $\IC_\mu^{ext}(\Pi) \ge 2\log n$.
    \end{proof}
\end{frame}
%%%%%%%%%%%%%%%%%%%%%%%%%%%%%%%%%%%%%%%%%%%%%%%%%%%%%%%%%%%%%%%%%%%%%%%%

\end{document}
