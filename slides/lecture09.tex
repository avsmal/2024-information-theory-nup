% !TeX spellcheck = en_US
\documentclass[aspectratio=169]{beamer}

\usetheme{default}

\setbeamertemplate{navigation symbols}{}

\usepackage{tikz}

\usepackage{amsmath,amssymb}

\newcommand{\pitem}{\pause\item}

\newtheorem{statement}{Statement}

\newcommand{\bits}{\{0,1\}}
\newcommand{\bitstr}{\bits^*}
\newcommand{\sshalf}{{\textstyle\frac12}}
\newcommand{\seqn}[2]{{#1}_1,{#1}_2,\dotsc,{#1}_{#2}}
\newcommand{\seqin}[3]{{#1}_{{#2}_1},{#1}_{{#2}_2},\dotsc,{#1}_{{#2}_{#3}}}
\newcommand{\IC}{\mathrm{IC}}
\newcommand{\poly}{\mathrm{poly}}
\newcommand{\Nat}{\mathbb{N}}

\DeclareMathOperator{\dom}{dom}
\DeclareMathOperator{\rank}{rank}
\DeclareMathOperator{\rng}{rng}
\DeclareMathOperator*{\E}{\mathbb{E}}

\DeclareMathOperator{\KL}{D_{KL}}
\DeclareMathOperator{\CE}{CE}
\newcommand{\midd}{\parallel}


\newenvironment{sol}{%
    \begin{block}{}%
        \setbeamercolor{block body}{bg=lightgray}%
    }{
\end{block}}

\AtBeginSection[]{
    \begin{frame}
        \vfill
        \centering
        \begin{beamercolorbox}[sep=8pt,center,shadow=true,rounded=true]{title}
            \usebeamerfont{title}\insertsectionhead\par%
        \end{beamercolorbox}
        \vfill
    \end{frame}
}


\title{Kolmogorov-Levin Theorem and\\ Applications of Kolmogorov Complexity}
\author{Alexander Smal at NUP}
\date{18.12.2024.\\ Lecture 9}

\begin{document}
%%%%%%%%%%%%%%%%%%%%%%%%%%%%%%%%%%%%%%%%%%%%%%%%%%%%%%%%%%%%%%%%%%%%%%%%
\begin{frame}[plain]
    \maketitle
\end{frame}
%%%%%%%%%%%%%%%%%%%%%%%%%%%%%%%%%%%%%%%%%%%%%%%%%%%%%%%%%%%%%%%%%%%%%%%%

%%%%%%%%%%%%%%%%%%%%%%%%%%%%%%%%%%%%%%%%%%%%%%%%%%%%%%%%%%%%%%%%%%%%%%%%
\begin{frame}{Kolmogorov-Levin Theorem}
\begin{theorem}[Kolmogorov-Levin]\label{thm:kolmogorov-levin}
    $|K(x,y) - K(x) - K(y \mid x)| \le O(\log K(x,y))$.
\end{theorem}

\begin{definition}
    \emph{Mutual information of $x$ and $y$:}
    \[
    \begin{aligned}
        I(x:y) &= K(y) - K(y \mid x),\\
        I(y:x) &= K(x) - K(x \mid y).
    \end{aligned}
    \]
\end{definition}

The Kolmogorov-Levin theorem states the symmetry of mutual information.
\[
I(x:y) = K(x) + K(y) - K(x,y) + O(\log K(x,y)) = I(y:x).
\]
\vspace{-5mm}
\begin{corollary}
    $|I(x:y) - I(y:x)| \le O(\log K(x,y))$.
\end{corollary}
\end{frame}
%%%%%%%%%%%%%%%%%%%%%%%%%%%%%%%%%%%%%%%%%%%%%%%%%%%%%%%%%%%%%%%%%%%%%%%%


%%%%%%%%%%%%%%%%%%%%%%%%%%%%%%%%%%%%%%%%%%%%%%%%%%%%%%%%%%%%%%%%%%%%%%%%
\begin{frame}{Kolmogorov-Levin Theorem: proof}
    The inequality `$\le$' has already been proven. It remains to prove `$\ge$'.
    \[
    \underbrace{K(x)}_{m} + \underbrace{K(y \mid x)}_l \le \underbrace{K(x,y)}_n + \underbrace{O(\log K(x,y))}_{O(\log n)}.
    \]
    Let $S = \{(a,b) \mid K(a,b) \le n\}$. Note that $(x,y) \in S$ and $|S| \le 2^{n+1}$.\\
    Consider $S_x = \{(x,b) \mid (x,b) \in S\}$.
    By definition, $(x,y) \in S_x$.
    \\We will show that
    \[
    l = K(y \mid x) \le \log |S_x| + O(\log n).
    \]
    We will enumerate the set $S$. During this enumeration, we will obtain points from $S_x$. To specify $y$, we need to indicate the index of $(x,y)$ in this enumeration. Additionally, to start such an enumeration, we need to know the number $n$. Thus,
    \[|S_x| \ge 2^{l - c \cdot \log n} \ge 2^{l'},\]
    where $l'$ is the nearest integer less than or equal to $l$, i.e., $l' = \lfloor l - c \cdot \log n \rfloor$.

\end{frame}
%%%%%%%%%%%%%%%%%%%%%%%%%%%%%%%%%%%%%%%%%%%%%%%%%%%%%%%%%%%%%%%%%%%%%%%%

%%%%%%%%%%%%%%%%%%%%%%%%%%%%%%%%%%%%%%%%%%%%%%%%%%%%%%%%%%%%%%%%%%%%%%%%
\begin{frame}{Kolmogorov-Levin Theorem: proof (cont.)}
    Let us look again at the enumeration of $S$. During the enumeration, we encounter ``heavy sections''—those in which the number of elements is at least $2^{l'}$. To specify the section $S_x$, we need to indicate its ordinal number in the enumeration of $S$ among all ``heavy sections''. Thus,
    \[
    m = K(x) \le \log(\text{\# heavy sections}) + O(\log n) + O(\log l').
    \]
    The number of heavy sections is at most $|S|/2^{l'}$.
    \[
    m = K(x) \le \log \frac{|S|}{2^{l'}} + O(\log n) = n - l + O(\log n).
    \]
    Thus, we obtain the statement of the theorem: $m + l \le n + O(\log n)$.

\end{frame}
%%%%%%%%%%%%%%%%%%%%%%%%%%%%%%%%%%%%%%%%%%%%%%%%%%%%%%%%%%%%%%%%%%%%%%%%

%%%%%%%%%%%%%%%%%%%%%%%%%%%%%%%%%%%%%%%%%%%%%%%%%%%%%%%%%%%%%%%%%%%%%%%%
\begin{frame}{Kolmogorov-Levin Theorem: final remark}
    Choose $n$ such that its binary representation is incompressible, i.e., $K(\bar n) = \log n + O(1)$. Take $x \in \{0,1\}^n$ such that $K(x \mid \bar n) = n + O(1)$. Then
\begin{itemize}
    \item\(
    I(\bar n : x) = K(x) - K(x \mid \bar n) = n + O(1) - (n + O(1)) = O(1),
    \)
    \item\(
    I(x : \bar n) = K(\bar n) - K(\bar n \mid x) = (\log n + O(1)) - O(1) = \log n + O(1).
    \)
\end{itemize}
Thus, it is not possible to reduce the logarithmic gap in the Kolmogorov-Levin theorem.
\end{frame}

%%%%%%%%%%%%%%%%%%%%%%%%%%%%%%%%%%%%%%%%%%%%%%%%%%%%%%%%%%%%%%%%%%%%%%%%
\begin{frame}{Infinity of the Set of Prime Numbers}
\begin{theorem}
    There are infinitely many prime numbers.
\end{theorem}
\begin{proof}
    \begin{itemize}
        \item Suppose there are only $m$ prime numbers: $\seqn{p}{m}$.
        \pitem Any integer $x$ is a product of powers of these primes:
        $
        x = p_1^{k_1}\cdot p_2^{k_2}\cdot\dotsm\cdot p_m^{k_m},
        $
        \pitem $x$ is uniquely defined by the set of powers $\seqn{k}{m}$.

        \pitem Each $k_i \le \log x$, so it can be described using $O(\log \log x)$ bits.

        \pitem $m$ is an absolute constant, so any $x$ can be described with $O(\log \log x)$ bits.

        \pitem There exist random $n$-bit numbers $x$ with complexity at least $n$.

        \pitem
        This leads to a contradiction:
        \[
        n \le K(x) \le O(\log \log x) = O(\log n).
        \]
    \end{itemize}
\end{proof}

\end{frame}
%%%%%%%%%%%%%%%%%%%%%%%%%%%%%%%%%%%%%%%%%%%%%%%%%%%%%%%%%%%%%%%%%%%%%%%%


%%%%%%%%%%%%%%%%%%%%%%%%%%%%%%%%%%%%%%%%%%%%%%%%%%%%%%%%%%%%%%%%%%%%%%%%
\begin{frame}{Method of Incompressible Objects}
    \begin{definition}
        \emph{Finite automata with multiple heads}: transition function is based on
        \begin{itemize}
            \item the internal state of the automata,
            \item the symbols under the heads,
        \end{itemize}
        returns
        \begin{itemize}
            \item the next state,
            \item and the indices of the heads to be moved,
            \item with at least one head moving at each step.
        \end{itemize}
    \end{definition}

    Class $\mathcal{L}_k$ is the class of languages recognized by finite automata with $k$ heads.

    \begin{theorem}\label{thm:kDFA}
        $\mathcal{L}_k \subsetneq \mathcal{L}_{k+1}$.
    \end{theorem}

\end{frame}
%%%%%%%%%%%%%%%%%%%%%%%%%%%%%%%%%%%%%%%%%%%%%%%%%%%%%%%%%%%%%%%%%%%%%%%%


%%%%%%%%%%%%%%%%%%%%%%%%%%%%%%%%%%%%%%%%%%%%%%%%%%%%%%%%%%%%%%%%%%%%%%%%
\begin{frame}[fragile]{Hard Language for $k$-head Automata}
    Consider the following family of languages over the alphabet $\{0,1,\#\}$
    \[
    A_n = \{w_1\#w_2\#\dotsb\#w_n\#w_n\#\dotsb\#w_1 \mid w_i \in \bitstr\},
    \]
    where $w_i \in \bitstr$, for all $i \in \{1,2,\dotsc,n\}$.

    For $n = 1$, the language $A_1 = \{w_1\#w_1\}$ requires two heads.\pause

    For $n = 3$, it can be recognized with our heads:
    \[
    \underset{\fbox{1}}{w_1}\#w_2\#w_3\#\underset{\fbox{2}}{w_3}\#\underset{\fbox{3}}{w_2}\#\underset{\fbox{4}}{w_1}.
    \]\pause
    But it can also be done with three heads (can you find a trick?):\pause
    \[
    \underset{\fbox{1}}{w_1}\#w_2\#\underset{\fbox{2}}{w_3}\#\underset{\fbox{3}}{w_3}\#w_2\#w_1.
    \]\pause

    Using the trick $k$ heads can recognize the language $A_n$ for $n \le (k-1) + \dotsb + 1$.
    \begin{lemma}
        $A_n$ is not recognized by a finite automata with $k$ heads if $n > \frac{k \cdot (k-1)}{2}$.
    \end{lemma}
\end{frame}
%%%%%%%%%%%%%%%%%%%%%%%%%%%%%%%%%%%%%%%%%%%%%%%%%%%%%%%%%%%%%%%%%%%%%%%%


%%%%%%%%%%%%%%%%%%%%%%%%%%%%%%%%%%%%%%%%%%%%%%%%%%%%%%%%%%%%%%%%%%%%%%%%
\begin{frame}{Proof: Protocol for Automata}
    A pair of heads $(i,j)$ \emph{inspects} $w_\ell$ if there is a step when the $i$-th head reads a symbol from the left copy of $w_\ell$ and the $j$-th head reads a symbol from the right copy of $w_\ell$.\medskip\pause

    For any $x \in A_n$ and for any pair $(i,j)$, there exists at most one block $w_\ell$ such that the pair $(i,j)$ inspects $w_\ell$. \medskip\pause

    If $n > k \cdot (k-1)/2$, then there will be a block that is not inspected by any pair of heads. Let us consider some $x \in A_n$ and assume that the block $w_\ell$ is not inspected.\medskip\pause

    The block $w_\ell$ is not inspected, so as long as some heads are in the left copy of $w_\ell$, no heads can be in the right copy of $w_\ell$.\medskip\pause

    Lets write down the \emph{protocol for the automata on the word $x$ with parameter $\ell$}.\\
    We record the state of the automata each time the following events occur:
    \begin{itemize}
        \item entry of a head into the copy of $w_\ell$,
        \item exit of a head from the copy of $w_\ell$.
    \end{itemize}
    The state of the automata will be described by its internal state and the positions of all heads. We denote such a protocol by $\pi(x,\ell)$.

\end{frame}
%%%%%%%%%%%%%%%%%%%%%%%%%%%%%%%%%%%%%%%%%%%%%%%%%%%%%%%%%%%%%%%%%%%%%%%%

%%%%%%%%%%%%%%%%%%%%%%%%%%%%%%%%%%%%%%%%%%%%%%%%%%%%%%%%%%%%%%%%%%%%%%%%
\begin{frame}{Proof: Protocols Defines Blocks}
    Suppose that for a specific
\[
    x = w_1\#w_2\#\dotsb\#w_\ell\#\dotsb\#w_n\#w_n\#\dotsb\#w_\ell\#\dotsb\#w_1,
\]
the finite automata does not inspect the block $\ell$.

Consider the input $x'$ with a different block $w'_\ell$:
\[
x' = w_1\#w_2\#\dotsb\#\alert{w'_\ell}\#\dotsb\#w_n\#w_n\#\dotsb\#\alert{w'_\ell}\#\dotsb\#w_1.
\]

If the protocols are equal, then the automata must also accept the input
\[
x'' = w_1\#w_2\#\dotsb\#w_\ell\#\dotsb\#w_n\#w_n\#\dotsb\#\alert{w'_\ell}\#\dotsb\#w_1.
\]
If some heads are in $w_\ell$, then the automata operates on $x''$ as it does on input $x$. If some heads are in $w'_\ell$, then the automata operates as it does on input $x'$. Therefore, it must accept $x'' \not\in A_n$.

Hence, different $w_\ell$ correspond for different protocols.
\end{frame}
%%%%%%%%%%%%%%%%%%%%%%%%%%%%%%%%%%%%%%%%%%%%%%%%%%%%%%%%%%%%%%%%%%%%%%%%

%%%%%%%%%%%%%%%%%%%%%%%%%%%%%%%%%%%%%%%%%%%%%%%%%%%%%%%%%%%%%%%%%%%%%%%%
\begin{frame}{Proof: Contradiction}
    Our observation can be rewritten as follows:
    \[
    K(w_\ell \mid w_1,\dots,w_{\ell-1},w_{\ell+1},\dotsc,w_n,\ell,\pi(x,\ell)) = O(1).
    \]

    Assume that all blocks have length $N$. Furthermore, we require that $x$ is incompressible, i.e., $K(x) = K(w_1,w_2,\dotsc,w_n) \ge n \cdot N$.
    Then,
    \[
    n \cdot N \le K(w_1,\dotsc,w_n) \le
    \underbrace{(n-1) \cdot N}_{\{w_i\}_{i \neq \ell}} +
    \underbrace{O(\log n)}_{\ell} +
    \underbrace{4 \cdot k \cdot O(k \log n N)}_{\text{complexity}\ \pi(x,\ell)}.
    \]
    As $N \to \infty$, we obtain a contradiction: $n \cdot N \le (n-1)N + O(k^2 \log n N)$.\medskip

The language $A_{\frac{k \cdot (k + 1)}{2}}$ belongs to $\mathcal{L}_{k+1}$ but does not belong to $\mathcal{L}_k$.

\end{frame}
%%%%%%%%%%%%%%%%%%%%%%%%%%%%%%%%%%%%%%%%%%%%%%%%%%%%%%%%%%%%%%%%%%%%%%%%



\end{document}
