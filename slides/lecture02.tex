% !TeX spellcheck = en_US
\documentclass[aspectratio=169]{beamer}
\usetheme{default}

\setbeamertemplate{navigation symbols}{}

\usepackage{amsmath,amssymb}


\newcommand{\pitem}{\pause\item}

\newcommand{\bits}{\{0,1\}}
\newcommand{\bitstr}{\bits^*}
\newcommand{\sshalf}{{\textstyle\frac12}}
\newcommand{\seqn}[2]{{#1}_1,{#1}_2,\dotsc,{#1}_{#2}}
\newcommand{\seqin}[3]{{#1}_{{#2}_1},{#1}_{{#2}_2},\dotsc,{#1}_{{#2}_{#3}}}
\newcommand{\IC}{\mathrm{IC}}
\newcommand{\poly}{\mathrm{poly}}
\newcommand{\Nat}{\mathbb{N}}

\DeclareMathOperator{\dom}{dom}
\DeclareMathOperator{\rank}{rank}
\DeclareMathOperator{\rng}{rng}
\DeclareMathOperator*{\E}{\mathbb{E}}

\newenvironment{sol}{%
\begin{block}{}%
\setbeamercolor{block body}{bg=lightgray}%
}{
\end{block}}

 \AtBeginSection[]{
       \begin{frame}
           \vfill
           \centering
           \begin{beamercolorbox}[sep=8pt,center,shadow=true,rounded=true]{title}
                 \usebeamerfont{title}\insertsectionhead\par%
               \end{beamercolorbox}
           \vfill
           \end{frame}
     }



\title{Information Theory}
\author{Alexander Smal at NUP}
\date{16.10.2024.\\ Lecture 2}
\begin{document}
%%%%%%%%%%%%%%%%%%%%%%%%%%%%%%%%%%%%%%%%%%%%%%%%%%%%%%%%%%%%%%%%%%%%%%%%
\begin{frame}[plain]
    \maketitle
\end{frame}
%%%%%%%%%%%%%%%%%%%%%%%%%%%%%%%%%%%%%%%%%%%%%%%%%%%%%%%%%%%%%%%%%%%%%%%%

\section{Probabilistic approach}
%%%%%%%%%%%%%%%%%%%%%%%%%%%%%%%%%%%%%%%%%%%%%%%%%%%%%%%%%%%%%%%%%%%%%%%%
\begin{frame}{Shannon Entropy}
Shannon entropy defines the amount of information \(H(\alpha)\) in the probability distribution for a random variable \(\alpha\).
\medskip\pause

Let \(\alpha\) take values from the set \(\{\seqn{a}{k}\}\) with probabilities \(\{\seqn{p}{k}\}\), where \(p_i \ge 0\) and \(\sum_i p_i = 1\).
\medskip\pause

We would like this definition to be consistent with Hartley's definition, i.e., the following ``boundary conditions'' hold:
\begin{itemize}
    \item if \(p_1 = \dotsb = p_k\), then \(H(\alpha) = \log k\),
    \item if \(p_1 = 1\), \(p_2 = \dotsb = p_k = 0\), then \(H(\alpha) = 0\).
\end{itemize}
\medskip\pause

We will look for \(H(\alpha)\) in the form of the expected value of the information we obtain from each outcome.
\[
H(\alpha) = \sum_i p_i \cdot \text{(information in $a_i$)}.
\]
\end{frame}
%%%%%%%%%%%%%%%%%%%%%%%%%%%%%%%%%%%%%%%%%%%%%%%%%%%%%%%%%%%%%%%%%%%%%%%%

%%%%%%%%%%%%%%%%%%%%%%%%%%%%%%%%%%%%%%%%%%%%%%%%%%%%%%%%%%%%%%%%%%%%%%%%
\begin{frame}{Shannon's Formula}
\begin{definition}[Shannon, 1948]
    The Shannon entropy of a random variable \(\alpha\) is given by
    \[
    H(\alpha) = \sum_{i=1}^k p_i \cdot \log \frac{1}{p_i}.
    \]
    (For continuity, we define \(0 \cdot \log \frac{0}{0} = 0\).)
    \end{definition}
    \medskip\pause

\end{frame}
%%%%%%%%%%%%%%%%%%%%%%%%%%%%%%%%%%%%%%%%%%%%%%%%%%%%%%%%%%%%%%%%%%%%%%%%

%%%%%%%%%%%%%%%%%%%%%%%%%%%%%%%%%%%%%%%%%%%%%%%%%%%%%%%%%%%%%%%%%%%%%%%%
\begin{frame}{Properties}
    \begin{lemma}\label{lm:entropy-properties}
    The following relations hold for Shannon entropy.
    \begin{itemize}
        \item \(H(\alpha) \ge 0\), and \(H(\alpha) = 0\) \(\iff\) the distribution of \(\alpha\) is degenerate.

        \item \(H(\alpha) \le \log k\), and \(H(\alpha) = \log k\) \(\iff\) the variable \(\alpha\) is uniformly distributed.

    \end{itemize}
\end{lemma}
\medskip\pause

To prove this, we will need the following theorem.

\begin{theorem}[Jensen's Inequality]
    Let the function \(f(x)\) be concave on some interval \(\mathcal{X}\) and let the numbers \(\seqn{q}{n} > 0\) be such that \(q_1 + \dotsb + q_n = 1\). Then for any \(\seqn{x}{n}\) from the interval \(\mathcal{X}\), the inequality holds:
    \[
    \sum_{i=1}^{n} q_{i} f(x_{i}) \leq f\left(\sum_{i=1}^{n} q_{i} x_{i}\right).
    \]
\end{theorem}



\end{frame}
%%%%%%%%%%%%%%%%%%%%%%%%%%%%%%%%%%%%%%%%%%%%%%%%%%%%%%%%%%%%%%%%%%%%%%%%

%%%%%%%%%%%%%%%%%%%%%%%%%%%%%%%%%%%%%%%%%%%%%%%%%%%%%%%%%%%%%%%%%%%%%%%%
\begin{frame}{Properties: proof}
        \begin{lemma}\label{lm:entropy-properties}
        The following relations hold for Shannon entropy.
        \begin{itemize}
            \item \(H(\alpha) \ge 0\), and \(H(\alpha) = 0\) \(\iff\) the distribution of \(\alpha\) is degenerate.

            \item \(H(\alpha) \le \log k\), and \(H(\alpha) = \log k\) \(\iff\) the variable \(\alpha\) is uniformly distributed.

        \end{itemize}
    \end{lemma}
    \begin{proof}
        The first property follows directly from the definition: each term in the sum \(H(\alpha)\) is non-negative and equals zero if and only if \(p_i = 0\) or \(p_i = 1\).

        To prove the second inequality, we will move everything to the left side and apply Jensen's inequality:
        \[
        H(\alpha) - \log k
        = \sum_{i=1}^k p_i \cdot \log \frac{1}{p_i} - \sum_{i=1}^k p_i \cdot \log k
        = \sum_{i=1}^k p_i \cdot \log \frac{1}{p_i k}
        \le \log\sum_{i=1}^k p_i \frac{1}{p_i k} = \log 1 = 0.
        \]
    \end{proof}
\end{frame}

%%%%%%%%%%%%%%%%%%%%%%%%%%%%%%%%%%%%%%%%%%%%%%%%%%%%%%%%%%%%%%%%%%%%%%%%
\begin{frame}{Entropy of a pair}
\begin{definition}
    The entropy of the joint distribution of the pair of random variables \(\alpha\) and \(\beta\) is \(H(\alpha, \beta)\).
\end{definition}

\begin{lemma}
    \begin{itemize}
        \item \(H(\alpha, \beta) \le H(\alpha) + H(\beta)\), equality iff the random variables are independent;
        \item \(H(\alpha) \le H(\alpha, \beta)\), equality iff \(\beta = f(\alpha)\).
    \end{itemize}
\end{lemma}
\end{frame}
%%%%%%%%%%%%%%%%%%%%%%%%%%%%%%%%%%%%%%%%%%%%%%%%%%%%%%%%%%%%%%%%%%%%%%%%


%%%%%%%%%%%%%%%%%%%%%%%%%%%%%%%%%%%%%%%%%%%%%%%%%%%%%%%%%%%%%%%%%%%%%%%%
\begin{frame}{Conditional Entropy}
\begin{definition}\label{def:cond-entropy1}
    The entropy of \(\alpha\) given \(\beta = b_j\) is defined as
    \[
    H(\alpha\mid\beta = b_j) = \sum_i \Pr[\alpha = a_i\mid \beta = b_j] \cdot
    \log\frac{1}{\Pr[\alpha = a_i\mid \beta = b_j]}.
    \]
\end{definition}
\vspace{-5mm}

\begin{definition}\label{def:cond-entropy2}
    The \emph{conditional (relative) entropy} of \(\alpha\) with respect to \(\beta\) is given by
    \[
    H(\alpha\mid\beta) = \sum_j \Pr[\beta = b_j] \cdot H(\alpha\mid \beta = b_j).
    \]
    In other words,
    \[
    H(\alpha\mid\beta) = \E\limits_{b_j \gets \beta}[H(\alpha\mid \beta = b_j)].
    \]
\end{definition}
\end{frame}
%%%%%%%%%%%%%%%%%%%%%%%%%%%%%%%%%%%%%%%%%%%%%%%%%%%%%%%%%%%%%%%%%%%%%%%%

%%%%%%%%%%%%%%%%%%%%%%%%%%%%%%%%%%%%%%%%%%%%%%%%%%%%%%%%%%%%%%%%%%%%%%%%
\begin{frame}{Properties}
\begin{lemma}
    The conditional entropy has the following properties:
    \begin{itemize}
        \item \(H(\alpha\mid\beta) \ge 0\).
        \item \(H(\alpha\mid\beta) = 0\) \(\iff\) \(\alpha\) is uniquely determined by \(\beta\).
        \item \(H(\alpha,\beta) = H(\beta) + H(\alpha\mid\beta) = H(\alpha) + H(\beta\mid\alpha)\).
    \end{itemize}
\end{lemma}
\medskip\pause

\begin{corollary}
    \(H(\alpha \mid \beta) \le H(\alpha)\), equality iff $\alpha$ and $\beta$ are independent.
\end{corollary}

\end{frame}
%%%%%%%%%%%%%%%%%%%%%%%%%%%%%%%%%%%%%%%%%%%%%%%%%%%%%%%%%%%%%%%%%%%%%%%%

%%%%%%%%%%%%%%%%%%%%%%%%%%%%%%%%%%%%%%%%%%%%%%%%%%%%%%%%%%%%%%%%%%%%%%%%
\begin{frame}{Mutual Information}
\begin{definition}
    The \emph{information in \(\alpha\) about the variable \(\beta\)} is defined as:
    \[
    I(\alpha:\beta) = H(\beta) - H(\beta\mid\alpha).
    \]
    It is also referred to as the \emph{mutual information of the random variables \(\alpha\) and \(\beta\)}.
\end{definition}

\begin{lemma}
    The following relations hold for mutual information:
    \begin{enumerate}
        \item \(I(\alpha:\beta) \le H(\alpha)\).
        \item \(I(\alpha:\beta) \le H(\beta)\).
        \item \(I(\alpha:\alpha) = H(\alpha)\).
        \item \(I(\alpha:\beta) = I(\beta:\alpha)\).
        \item \(I(\alpha:\beta) = H(\alpha) + H(\beta) - H(\alpha,\beta)\).
    \end{enumerate}
\end{lemma}
\end{frame}
%%%%%%%%%%%%%%%%%%%%%%%%%%%%%%%%%%%%%%%%%%%%%%%%%%%%%%%%%%%%%%%%%%%%%%%%


%%%%%%%%%%%%%%%%%%%%%%%%%%%%%%%%%%%%%%%%%%%%%%%%%%%%%%%%%%%%%%%%%%%%%%%%
\begin{frame}{Conditional Mutual Information}
\begin{definition}
    The \emph{information in \(\alpha\) about \(\beta\) given \(\gamma\)} is define as follows:
    \begin{enumerate}
        \item \(I(\alpha:\beta\mid\gamma) = H(\beta\mid\gamma) - H(\beta\mid\alpha,\gamma)\).
        \item \(I(\alpha:\beta\mid\gamma) = \sum_\ell I(\alpha:\beta \mid \gamma = c_\ell) \cdot \Pr[\gamma = c_\ell]\).
        \item \(I(\alpha:\beta\mid\gamma) = H(\alpha\mid\gamma) + H(\beta\mid\gamma) - H(\alpha,\beta\mid\gamma)\).
        \item \(I(\alpha:\beta\mid\gamma) = H(\alpha,\gamma) + H(\beta,\gamma) - H(\alpha,\beta,\gamma) - H(\gamma)\).
    \end{enumerate}
\end{definition}

\begin{lemma}
    All definitions of conditional mutual information are equivalent.
\end{lemma}

\begin{lemma}[Chain rule for mutual information]
    The following relations hold:
    \begin{enumerate}
        \item \(I((\alpha,\beta) : \gamma) = I(\alpha : \gamma) + I(\beta: \gamma \mid \alpha)\).
        \item \(I((\alpha,\beta) : \gamma \mid \delta) = I(\alpha : \gamma \mid \delta) + I(\beta: \gamma \mid \alpha, \delta)\).
    \end{enumerate}
\end{lemma}
\end{frame}
%%%%%%%%%%%%%%%%%%%%%%%%%%%%%%%%%%%%%%%%%%%%%%%%%%%%%%%%%%%%%%%%%%%%%%%%

\end{document}


%%%%%%%%%%%%%%%%%%%%%%%%%%%%%%%%%%%%%%%%%%%%%%%%%%%%%%%%%%%%%%%%%%%%%%%%
\begin{frame}{}
    .
\end{frame}
%%%%%%%%%%%%%%%%%%%%%%%%%%%%%%%%%%%%%%%%%%%%%%%%%%%%%%%%%%%%%%%%%%%%%%%%

%%%%%%%%%%%%%%%%%%%%%%%%%%%%%%%%%%%%%%%%%%%%%%%%%%%%%%%%%%%%%%%%%%%%%%%%
\begin{frame}{}
    .
\end{frame}
%%%%%%%%%%%%%%%%%%%%%%%%%%%%%%%%%%%%%%%%%%%%%%%%%%%%%%%%%%%%%%%%%%%%%%%%

%%%%%%%%%%%%%%%%%%%%%%%%%%%%%%%%%%%%%%%%%%%%%%%%%%%%%%%%%%%%%%%%%%%%%%%%
\begin{frame}{}
    .
\end{frame}
%%%%%%%%%%%%%%%%%%%%%%%%%%%%%%%%%%%%%%%%%%%%%%%%%%%%%%%%%%%%%%%%%%%%%%%%

%%%%%%%%%%%%%%%%%%%%%%%%%%%%%%%%%%%%%%%%%%%%%%%%%%%%%%%%%%%%%%%%%%%%%%%%
\begin{frame}{}
    .
\end{frame}
%%%%%%%%%%%%%%%%%%%%%%%%%%%%%%%%%%%%%%%%%%%%%%%%%%%%%%%%%%%%%%%%%%%%%%%%

%%%%%%%%%%%%%%%%%%%%%%%%%%%%%%%%%%%%%%%%%%%%%%%%%%%%%%%%%%%%%%%%%%%%%%%%
\begin{frame}{}
    .
\end{frame}
%%%%%%%%%%%%%%%%%%%%%%%%%%%%%%%%%%%%%%%%%%%%%%%%%%%%%%%%%%%%%%%%%%%%%%%%




\end{document}
